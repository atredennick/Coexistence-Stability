\documentclass[12pt,]{article}
\usepackage{lmodern}
\usepackage{amssymb,amsmath}
\usepackage{ifxetex,ifluatex}
\usepackage{fixltx2e} % provides \textsubscript
\ifnum 0\ifxetex 1\fi\ifluatex 1\fi=0 % if pdftex
  \usepackage[T1]{fontenc}
  \usepackage[utf8]{inputenc}
\else % if luatex or xelatex
  \ifxetex
    \usepackage{mathspec}
    \usepackage{xltxtra,xunicode}
  \else
    \usepackage{fontspec}
  \fi
  \defaultfontfeatures{Mapping=tex-text,Scale=MatchLowercase}
  \newcommand{\euro}{€}
\fi
% use upquote if available, for straight quotes in verbatim environments
\IfFileExists{upquote.sty}{\usepackage{upquote}}{}
% use microtype if available
\IfFileExists{microtype.sty}{%
\usepackage{microtype}
\UseMicrotypeSet[protrusion]{basicmath} % disable protrusion for tt fonts
}{}
\usepackage[margin=1in]{geometry}
\ifxetex
  \usepackage[setpagesize=false, % page size defined by xetex
              unicode=false, % unicode breaks when used with xetex
              xetex]{hyperref}
\else
  \usepackage[unicode=true]{hyperref}
\fi
\hypersetup{breaklinks=true,
            bookmarks=true,
            pdfauthor={},
            pdftitle={},
            colorlinks=true,
            citecolor=blue,
            urlcolor=blue,
            linkcolor=magenta,
            pdfborder={0 0 0}}
\urlstyle{same}  % don't use monospace font for urls
\setlength{\parindent}{0pt}
\setlength{\parskip}{6pt plus 2pt minus 1pt}
\setlength{\emergencystretch}{3em}  % prevent overfull lines
\setcounter{secnumdepth}{0}

%%% Use protect on footnotes to avoid problems with footnotes in titles
\let\rmarkdownfootnote\footnote%
\def\footnote{\protect\rmarkdownfootnote}

%%% Change title format to be more compact
\usepackage{titling}

% Create subtitle command for use in maketitle
\newcommand{\subtitle}[1]{
  \posttitle{
    \begin{center}\large#1\end{center}
    }
}

\setlength{\droptitle}{-2em}
  \title{}
  \pretitle{\vspace{\droptitle}}
  \posttitle{}
  \author{}
  \preauthor{}\postauthor{}
  \date{}
  \predate{}\postdate{}

\usepackage{lineno}
\linenumbers
\usepackage{setspace}
\doublespacing


\begin{document}

\maketitle


\section{How coexistence mechanisms mediate temporal
stability}\label{how-coexistence-mechanisms-mediate-temporal-stability}

\subsubsection{Andrew T. Tredennick, Peter B. Adler, and Frederick R.
Adler}\label{andrew-t.-tredennick-peter-b.-adler-and-frederick-r.-adler}

\emph{Andrew T. Tredennick
(\href{mailto:atredenn@gmail.com}{\href{mailto:atredenn@gmail.com}{\nolinkurl{atredenn@gmail.com}}}),
Department of Wildland Resources and the Ecology Center, Utah State
University, Logan, UT}

\emph{Peter B. Adler, Department of Wildland Resources and the Ecology
Center, Utah State University, Logan, UT}

\emph{Frederick R. Adler, Departments of Biology and Mathematics,
University of Utah, Salt Lake City, UT}

\subsection{Introduction}\label{introduction}

It is now clear that species richness tends to increase the temporal
stability of biomass production in competitive plant communities. It is
even becoming clear that a few core mechanisms related to compensatory
dynamics among species through temporal or functional complementarity
can explain the positive relationship between diversity and stability.
What remains unclear is how the coexistence mechanism that maintains
species richness influences the strength of compensatory dynamics.
Theoretical studies based on Lotka-Volterra models have given us insight
into the mechanisms that promote stability in communities where species
coexist by fluctuation-independent mechanisms. A remaining challenge is
to explore the importance of proposed mechanisms behind
diversity-stability relationships in communities where species
coexistence is maintained by temporal variability. Specifically, we seek
to understand the mechanisms that promote temporal stability in
communities where species coexistence is achieved because of temporal
variability.

To that end, we will analyze a general consumer-resource model under
different fluctuation-dependent coexistence assmptions. Our starting
point is a model of two plant consumers and one resource (e.g., soil
moisture or nitrogen). We will focus on three cases of species
coexistence:

\begin{enumerate}
  \item Relative nonlinearty
  \item Temporal storage effect
  \item A combination of both mechanisms
\end{enumerate}

Each scenario requires different model assumptions and structure, so we
will describe each in turn. Although the structure may change slightly
to incorporate different coexistence mechanisms, the strength of our
approach lies in the similarities among the models since we work under a
unified consumer-resource framework.

\subsection{Model description and
analysis}\label{model-description-and-analysis}

\subsubsection{A general consumer-resource
model}\label{a-general-consumer-resource-model}

We start with a general consumer-resource model where the consumer can
be in one of two-states: a dormant state \(D\) and a live state \(N\)
(Fig. 1). Transitions between \(N\) and \(D\) occur at discrete
intervals \(T\), so our model is formulated as ``pulsed differential
equations" (Pachepsky et al. 2008, Mailleret and Lemesle 2009). For
clarity we refer to \(T\) as years and the growing time between years,
\(\tau\), as seasons. Seasonal (within-year) dynamics are modeled as
three differential equations:

\begin{align}
\frac{\text{d}D_{i}}{\text{d}\tau} &= -(m_{D,i}D_{i})\\
\frac{\text{d}N_{i}}{\text{d}\tau} &= N_{i}[f_{i}(R) - m_{N,i}]\\
\frac{\text{d}R_{i}}{\text{d}\tau} &= - \sum\limits_{i=1,2}f_{i}(R)N_{i}
\end{align}

where \(i\) denotes species, \(D\) is the dormant (long-lived) biomass
state, \(N\) is the living biomass (fast-growing, shorter-lived) state,
and \(m\)s are biomass loss rates. The growth rate of living biomass is
a resource-dependent Hill function,
\(f_{i}(R) = r_{i}R^{\alpha_{i}} / (\beta_{i}^{\alpha_{i}}+R^{\alpha_{i}})\).
Resource depletion is equal to the sum of each species' consumption,
\(\sum_{i=1,2}f_{i}(R)N_{i}\). Note that since transitions between \(N\)
and \(D\) are pulsed, only biomass loss occurs throughout the season for
\(D\).

At the beginning of each season we start with initial conditions defined
as \(V_{t}\), \(W_{t}\), and \(Z_{t}\) for the dormant state, the live
state, and the resource, respectively. So for each season, Eqs. 1-3 are
solved given the initial conditions:

\begin{align}
  D_{i}(0) &= V_{i,t} \\
  N_{i}(0) &= W_{i,t} \\
  R(0) &= Z_{t}
\end{align}

The consumers transition between \(N\) and \(D\) instantaneously between
years. We assume resource density does not change between years. So, at
the yearly transition:

\begin{align}
  V_{i,t+1} &= [N_{i}(T^-)+D_{i}(T^-)](1-g_{t}) \\
  W_{i,t+1} &= [N_{i}(T^-)+D_{i}(T^-)]g_{t} \\
  Z_{t+1} &= R(T^-) + R(T^+)
\end{align}

where \(D(T^-)\), \(N(T^-)\), and \(R(T^-)\) are the densities of each
state at the end of the year and \(g\) is a time-fluctuating activation
rate that regulates how much dormant biomass is converted to
growing-season live biomass each year. \(R(T^+)\) is a randomly
generated resource pulse from a log-normal distribution with mean
\(\mu_{R}\) and standard deviation \(\sigma_{R}\). Our formulation
assumes that at the end of each season all accumulated living biomass
{[}\(N(T^-)\){]} is converted to dormant biomass.

\begin{center}
\begin{table}
\caption{Definition of model parameters.}
\begin{tabular}{l l}
\hline
Parameter & Definition \\
\hline
$m_{D}$ & dormant state mortality rate \\
$r$ & live state maximum resource uptake rate \\
$K$ & live state half-saturation constant for resource uptake rate \\
$m_{N}$ & live state mortality rate \\
$a$ & resource turnover rate \\
$S$ & resource supply rate \\
$g$ & dormant-to-live biomass transition fraction \\ 
\hline
\end{tabular}
\end{table}
\end{center}

\subsubsection{Implementing the storage
effect}\label{implementing-the-storage-effect}

To make this a ``storage-effect" model, we need to satisfy three
conditions: (1) the organisms must have a mechanism for persistence
under unfavorable conditions, (2) species must respond differently to
environmental conditions, and (3) the effects of competition on a
species must be more strongly negative in good years relative to
unfavorable years. Our model meets condition 1 because we include a
dormant stage with very low death rates. We satisfy condition 2 with our
model whenever \(g\) is not perfectly correlated between species.
Lastly, our model meets condition 3 because condition 2 partitions
intraspecific and interspecific competition into different years. Thus,
during a high \(g\) year for one species, resource uptake is also
inherently high for that species, which increases intraspecific
competition relative to interspecific competition. So, given adequate
variability in \(g\), the inferior competitor (species with lower \(r\))
can persist.

Following ({\textbf{???}}), we generated sequences of (un)correlated
dormant-to-live state transition rates (\(g\)) for each species by
drawing from multivariate normal distributions with mean 0 and a
variance-covariance matrix (\(\Sigma_g\)) of

\begin{align}
\Sigma_g = 
\begin{bmatrix}
\sigma^2_{E} & \rho\sigma^2_{E} \\
\rho\sigma^2_{E} & \sigma^2_{E}
\end{bmatrix}
\end{align}

\noindent{}where \(\sigma^2_{E}\) is the variance and \(\rho\) is the
correlation between between the two species' transition rates. For
environmental variability, here induced as variability in \(g\), to
promote coexistence via the storage effect, \(\rho\) must be less than
1. The inferior competitor has the strongest potential to persist when
\(\rho=-1\) (perfectly uncorrelated transition rates).

\subsubsection{Implementing relative
nonlinearity}\label{implementing-relative-nonlinearity}

When considering consumer-resource dynamics, species coexistence by
relative nonlinearity requires that each species has different nonlinear
responses to resource availability, and resource availability must
fluctuate through time. In a constant resource environment, the species
with the lowest R* will always exclude the other species. So we can
compare this model to the storage effect model, we still allow the
germination rate \emph{g} to vary, but both species are perfectly
correlate -- that is, \(\rho=1\).

\subsection{Results}\label{results}

\subsubsection{Storage effect model}\label{storage-effect-model}

\pagebreak{}

\subsection*{References}\label{references}
\addcontentsline{toc}{subsection}{References}

Mailleret, L., and V. Lemesle. 2009. A note on semi-discrete modelling
in the life sciences. Philosophical transactions. Series A,
Mathematical, physical, and engineering sciences 367:4779--4799.

Pachepsky, E., R. M. Nisbet, and W. W. Murdoch. 2008. Between discrete
and continuous: Consumer-resource dynamics with synchronized
reproduction. Ecology 89:280--288.

\end{document}
