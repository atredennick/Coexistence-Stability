\documentclass[11pt,]{article}
\usepackage{lmodern}
\usepackage{amssymb,amsmath}
\usepackage{ifxetex,ifluatex}
\usepackage{fixltx2e} % provides \textsubscript
\ifnum 0\ifxetex 1\fi\ifluatex 1\fi=0 % if pdftex
  \usepackage[T1]{fontenc}
  \usepackage[utf8]{inputenc}
\else % if luatex or xelatex
  \ifxetex
    \usepackage{mathspec}
  \else
    \usepackage{fontspec}
  \fi
  \defaultfontfeatures{Ligatures=TeX,Scale=MatchLowercase}
\fi
% use upquote if available, for straight quotes in verbatim environments
\IfFileExists{upquote.sty}{\usepackage{upquote}}{}
% use microtype if available
\IfFileExists{microtype.sty}{%
\usepackage{microtype}
\UseMicrotypeSet[protrusion]{basicmath} % disable protrusion for tt fonts
}{}
\usepackage[margin=1in]{geometry}
\usepackage{hyperref}
\PassOptionsToPackage{usenames,dvipsnames}{color} % color is loaded by hyperref
\hypersetup{unicode=true,
            colorlinks=true,
            linkcolor=black,
            citecolor=Blue,
            urlcolor=black,
            breaklinks=true}
\urlstyle{same}  % don't use monospace font for urls
\usepackage{color}
\usepackage{fancyvrb}
\newcommand{\VerbBar}{|}
\newcommand{\VERB}{\Verb[commandchars=\\\{\}]}
\DefineVerbatimEnvironment{Highlighting}{Verbatim}{commandchars=\\\{\}}
% Add ',fontsize=\small' for more characters per line
\usepackage{framed}
\definecolor{shadecolor}{RGB}{248,248,248}
\newenvironment{Shaded}{\begin{snugshade}}{\end{snugshade}}
\newcommand{\KeywordTok}[1]{\textcolor[rgb]{0.13,0.29,0.53}{\textbf{{#1}}}}
\newcommand{\DataTypeTok}[1]{\textcolor[rgb]{0.13,0.29,0.53}{{#1}}}
\newcommand{\DecValTok}[1]{\textcolor[rgb]{0.00,0.00,0.81}{{#1}}}
\newcommand{\BaseNTok}[1]{\textcolor[rgb]{0.00,0.00,0.81}{{#1}}}
\newcommand{\FloatTok}[1]{\textcolor[rgb]{0.00,0.00,0.81}{{#1}}}
\newcommand{\ConstantTok}[1]{\textcolor[rgb]{0.00,0.00,0.00}{{#1}}}
\newcommand{\CharTok}[1]{\textcolor[rgb]{0.31,0.60,0.02}{{#1}}}
\newcommand{\SpecialCharTok}[1]{\textcolor[rgb]{0.00,0.00,0.00}{{#1}}}
\newcommand{\StringTok}[1]{\textcolor[rgb]{0.31,0.60,0.02}{{#1}}}
\newcommand{\VerbatimStringTok}[1]{\textcolor[rgb]{0.31,0.60,0.02}{{#1}}}
\newcommand{\SpecialStringTok}[1]{\textcolor[rgb]{0.31,0.60,0.02}{{#1}}}
\newcommand{\ImportTok}[1]{{#1}}
\newcommand{\CommentTok}[1]{\textcolor[rgb]{0.56,0.35,0.01}{\textit{{#1}}}}
\newcommand{\DocumentationTok}[1]{\textcolor[rgb]{0.56,0.35,0.01}{\textbf{\textit{{#1}}}}}
\newcommand{\AnnotationTok}[1]{\textcolor[rgb]{0.56,0.35,0.01}{\textbf{\textit{{#1}}}}}
\newcommand{\CommentVarTok}[1]{\textcolor[rgb]{0.56,0.35,0.01}{\textbf{\textit{{#1}}}}}
\newcommand{\OtherTok}[1]{\textcolor[rgb]{0.56,0.35,0.01}{{#1}}}
\newcommand{\FunctionTok}[1]{\textcolor[rgb]{0.00,0.00,0.00}{{#1}}}
\newcommand{\VariableTok}[1]{\textcolor[rgb]{0.00,0.00,0.00}{{#1}}}
\newcommand{\ControlFlowTok}[1]{\textcolor[rgb]{0.13,0.29,0.53}{\textbf{{#1}}}}
\newcommand{\OperatorTok}[1]{\textcolor[rgb]{0.81,0.36,0.00}{\textbf{{#1}}}}
\newcommand{\BuiltInTok}[1]{{#1}}
\newcommand{\ExtensionTok}[1]{{#1}}
\newcommand{\PreprocessorTok}[1]{\textcolor[rgb]{0.56,0.35,0.01}{\textit{{#1}}}}
\newcommand{\AttributeTok}[1]{\textcolor[rgb]{0.77,0.63,0.00}{{#1}}}
\newcommand{\RegionMarkerTok}[1]{{#1}}
\newcommand{\InformationTok}[1]{\textcolor[rgb]{0.56,0.35,0.01}{\textbf{\textit{{#1}}}}}
\newcommand{\WarningTok}[1]{\textcolor[rgb]{0.56,0.35,0.01}{\textbf{\textit{{#1}}}}}
\newcommand{\AlertTok}[1]{\textcolor[rgb]{0.94,0.16,0.16}{{#1}}}
\newcommand{\ErrorTok}[1]{\textcolor[rgb]{0.64,0.00,0.00}{\textbf{{#1}}}}
\newcommand{\NormalTok}[1]{{#1}}
\usepackage{graphicx,grffile}
\makeatletter
\def\maxwidth{\ifdim\Gin@nat@width>\linewidth\linewidth\else\Gin@nat@width\fi}
\def\maxheight{\ifdim\Gin@nat@height>\textheight\textheight\else\Gin@nat@height\fi}
\makeatother
% Scale images if necessary, so that they will not overflow the page
% margins by default, and it is still possible to overwrite the defaults
% using explicit options in \includegraphics[width, height, ...]{}
\setkeys{Gin}{width=\maxwidth,height=\maxheight,keepaspectratio}
\IfFileExists{parskip.sty}{%
\usepackage{parskip}
}{% else
\setlength{\parindent}{0pt}
\setlength{\parskip}{6pt plus 2pt minus 1pt}
}
\setlength{\emergencystretch}{3em}  % prevent overfull lines
\providecommand{\tightlist}{%
  \setlength{\itemsep}{0pt}\setlength{\parskip}{0pt}}
\setcounter{secnumdepth}{5}
% Redefines (sub)paragraphs to behave more like sections
\ifx\paragraph\undefined\else
\let\oldparagraph\paragraph
\renewcommand{\paragraph}[1]{\oldparagraph{#1}\mbox{}}
\fi
\ifx\subparagraph\undefined\else
\let\oldsubparagraph\subparagraph
\renewcommand{\subparagraph}[1]{\oldsubparagraph{#1}\mbox{}}
\fi

%%% Use protect on footnotes to avoid problems with footnotes in titles
\let\rmarkdownfootnote\footnote%
\def\footnote{\protect\rmarkdownfootnote}

%%% Change title format to be more compact
\usepackage{titling}

% Create subtitle command for use in maketitle
\newcommand{\subtitle}[1]{
  \posttitle{
    \begin{center}\large#1\end{center}
    }
}

\setlength{\droptitle}{-2em}
  \title{}
  \pretitle{\vspace{\droptitle}}
  \posttitle{}
  \author{}
  \preauthor{}\postauthor{}
  \date{}
  \predate{}\postdate{}

\usepackage{lineno}
\linenumbers
\usepackage{setspace}
\usepackage{todonotes}
\onehalfspacing
\usepackage{rotating}
\usepackage{color, soul}
\usepackage[font={small},labelfont={sf,bf},labelsep=space]{caption}
\usepackage{tikz}

\begin{document}

\newcommand{\tikzcircle}[2][red,fill=red]{\tikz[baseline=-0.5ex]\draw[#1,radius=#2] (0,0) circle ;}
\renewcommand\linenumberfont{\normalfont\tiny\sffamily\color{gray}}
\renewcommand\thefigure{S1-\arabic{figure}}  
\renewcommand\thetable{S1-\arabic{table}}  
\renewcommand\thesection{Section S1.\arabic{section}}

\begin{center}
\textbf{\Large{Supporting Information Appendix S1}} \\
A.T. Tredennick, P.B. Adler, \& F.R. Adler, ``The relationship between species richness and...'' \\
\emph{Ecology Letters}
\end{center}

\section{R Code for Consumer-Resource Model}

Below is the R code for our model function, which is represented
mathematically in the main text in Equations 1-4. The same code, along
with all the code to reproduce our results, has been archived on
Figshare (link) and is available on GitHub
(\url{http://github.com/atredennick/Coexistence-Stability/releases}).

\begin{Shaded}
\begin{Highlighting}[]
\NormalTok{simulate_model <-}\StringTok{ }\NormalTok{function(seasons, days_to_track, Rmu, }
                           \NormalTok{Rsd_annual, sigE, rho, }
                           \NormalTok{alpha1, alpha2, alpha3, alpha4,}
                           \NormalTok{eta1, eta2, eta3, eta4,}
                           \NormalTok{nu, r1, r2, r3, r4,}
                           \NormalTok{a1, a2, a3, a4,}
                           \NormalTok{b1, b2, b3, b4,}
                           \NormalTok{eps1, eps2, eps3, eps4,}
                           \NormalTok{D1, D2, D3, D4,}
                           \NormalTok{N1, N2, N3, N4, R) \{}
  
  \KeywordTok{require}\NormalTok{(}\StringTok{'deSolve'}\NormalTok{) }\CommentTok{# for solving continuous differential equations}
  \KeywordTok{require}\NormalTok{(}\StringTok{'mvtnorm'}\NormalTok{) }\CommentTok{# for multivariate normal distribution functions}
  
  \NormalTok{##  Assign parameter values to appropriate lists}
  \NormalTok{DNR <-}\StringTok{ }\KeywordTok{c}\NormalTok{(}\DataTypeTok{D=}\KeywordTok{c}\NormalTok{(D1,D2,D3,D4),   }\CommentTok{# initial dormant state abundance}
           \DataTypeTok{N=}\KeywordTok{c}\NormalTok{(N1,N2,N3,N4),   }\CommentTok{# initial live state abundance}
           \DataTypeTok{R=}\NormalTok{R)                }\CommentTok{# initial resource level}
  
  \NormalTok{parms <-}\StringTok{ }\KeywordTok{list} \NormalTok{(}
    \DataTypeTok{r   =} \KeywordTok{c}\NormalTok{(r1,r2,r3,r4),          }\CommentTok{# max growth rate for each species}
    \DataTypeTok{a   =} \KeywordTok{c}\NormalTok{(a1,a2,a3,a4),          }\CommentTok{# rate parameter for Hill function }
    \DataTypeTok{b   =} \KeywordTok{c}\NormalTok{(b1,b2,b3,b4),          }\CommentTok{# shape parameter for Hill function}
    \DataTypeTok{eps =} \KeywordTok{c}\NormalTok{(eps1,eps2,eps3,eps4)   }\CommentTok{# resource-to-biomass efficiency}
  \NormalTok{)}
  
  
  \NormalTok{####}
  \NormalTok{####  Sub-Model functions ----------------------------------------------------}
  \NormalTok{####}
  \NormalTok{## Continuous model (Equations 1-2)}
  \NormalTok{updateNR <-}\StringTok{ }\NormalTok{function(t, NR, parms)\{}
    \KeywordTok{with}\NormalTok{(}\KeywordTok{as.list}\NormalTok{(}\KeywordTok{c}\NormalTok{(NR, parms)), \{}
      \NormalTok{dN1dt =}\StringTok{ }\NormalTok{N1*eps[}\DecValTok{1}\NormalTok{]*(}\KeywordTok{uptake_R}\NormalTok{(r[}\DecValTok{1}\NormalTok{], R, a[}\DecValTok{1}\NormalTok{], b[}\DecValTok{1}\NormalTok{]))}
      \NormalTok{dN2dt =}\StringTok{ }\NormalTok{N2*eps[}\DecValTok{2}\NormalTok{]*(}\KeywordTok{uptake_R}\NormalTok{(r[}\DecValTok{2}\NormalTok{], R, a[}\DecValTok{2}\NormalTok{], b[}\DecValTok{2}\NormalTok{]))}
      \NormalTok{dN3dt =}\StringTok{ }\NormalTok{N3*eps[}\DecValTok{3}\NormalTok{]*(}\KeywordTok{uptake_R}\NormalTok{(r[}\DecValTok{3}\NormalTok{], R, a[}\DecValTok{3}\NormalTok{], b[}\DecValTok{3}\NormalTok{]))}
      \NormalTok{dN4dt =}\StringTok{ }\NormalTok{N4*eps[}\DecValTok{4}\NormalTok{]*(}\KeywordTok{uptake_R}\NormalTok{(r[}\DecValTok{4}\NormalTok{], R, a[}\DecValTok{4}\NormalTok{], b[}\DecValTok{4}\NormalTok{]))}
      \NormalTok{dRdt  =}\StringTok{ }\NormalTok{-}\DecValTok{1} \NormalTok{*}\StringTok{ }\NormalTok{(dN1dt/eps[}\DecValTok{1}\NormalTok{] +}\StringTok{ }\NormalTok{dN2dt/eps[}\DecValTok{2}\NormalTok{] +}\StringTok{ }\NormalTok{dN3dt/eps[}\DecValTok{3}\NormalTok{] +}\StringTok{ }\NormalTok{dN4dt/eps[}\DecValTok{4}\NormalTok{])}
      \KeywordTok{list}\NormalTok{(}\KeywordTok{c}\NormalTok{(dN1dt, dN2dt, dN3dt, dN4dt, dRdt)) }\CommentTok{# output as list}
    \NormalTok{\})}
  \NormalTok{\} }\CommentTok{# end continuous function}
  
  \NormalTok{## Discrete model (Equations 3-4)}
  \NormalTok{update_DNR <-}\StringTok{ }\NormalTok{function(t, DNR, gammas,}
                         \NormalTok{alpha1, alpha2, alpha3, alpha4,}
                         \NormalTok{eta1, eta2, eta3, eta4) \{}
    \KeywordTok{with} \NormalTok{(}\KeywordTok{as.list}\NormalTok{(DNR),\{}
      \NormalTok{g1    <-}\StringTok{ }\NormalTok{gammas[}\DecValTok{1}\NormalTok{]}
      \NormalTok{g2    <-}\StringTok{ }\NormalTok{gammas[}\DecValTok{2}\NormalTok{]}
      \NormalTok{g3    <-}\StringTok{ }\NormalTok{gammas[}\DecValTok{3}\NormalTok{]}
      \NormalTok{g4    <-}\StringTok{ }\NormalTok{gammas[}\DecValTok{4}\NormalTok{]}
      \NormalTok{D1new <-}\StringTok{ }\NormalTok{alpha1*N1 +}\StringTok{ }\NormalTok{D1*(}\DecValTok{1}\NormalTok{-g1)*(}\DecValTok{1}\NormalTok{-eta1)}
      \NormalTok{D2new <-}\StringTok{ }\NormalTok{alpha2*N2 +}\StringTok{ }\NormalTok{D2*(}\DecValTok{1}\NormalTok{-g2)*(}\DecValTok{1}\NormalTok{-eta2)}
      \NormalTok{D3new <-}\StringTok{ }\NormalTok{alpha3*N3 +}\StringTok{ }\NormalTok{D3*(}\DecValTok{1}\NormalTok{-g3)*(}\DecValTok{1}\NormalTok{-eta3)}
      \NormalTok{D4new <-}\StringTok{ }\NormalTok{alpha4*N4 +}\StringTok{ }\NormalTok{D4*(}\DecValTok{1}\NormalTok{-g4)*(}\DecValTok{1}\NormalTok{-eta4)}
      \NormalTok{N1new <-}\StringTok{ }\NormalTok{g1*(D1+(alpha1*N1))*(}\DecValTok{1}\NormalTok{-eta1)}
      \NormalTok{N2new <-}\StringTok{ }\NormalTok{g2*(D2+(alpha2*N2))*(}\DecValTok{1}\NormalTok{-eta2)}
      \NormalTok{N3new <-}\StringTok{ }\NormalTok{g3*(D3+(alpha3*N3))*(}\DecValTok{1}\NormalTok{-eta3)}
      \NormalTok{N4new <-}\StringTok{ }\NormalTok{g4*(D4+(alpha4*N4))*(}\DecValTok{1}\NormalTok{-eta4)}
      \NormalTok{Rnew  <-}\StringTok{ }\NormalTok{Rvector[t]}
      \KeywordTok{return}\NormalTok{(}\KeywordTok{c}\NormalTok{(D1new, D2new, D3new, D4new, N1new, N2new, N3new, N4new, Rnew))}
    \NormalTok{\})}
  \NormalTok{\}}
  
  \NormalTok{##  Resource uptake function (Hill function)}
  \NormalTok{uptake_R <-}\StringTok{ }\NormalTok{function(r, R, a, b) \{}
    \KeywordTok{return}\NormalTok{((r*R^a) /}\StringTok{ }\NormalTok{(b^a +}\StringTok{ }\NormalTok{R^a))}
  \NormalTok{\}}
  
  \NormalTok{##  Generate germination fractions}
  \NormalTok{getG <-}\StringTok{ }\NormalTok{function(sigE, rho, nTime, num_spp) \{}
    \NormalTok{varcov       <-}\StringTok{ }\KeywordTok{matrix}\NormalTok{(}\KeywordTok{rep}\NormalTok{(rho*sigE,num_spp*}\DecValTok{2}\NormalTok{), num_spp, num_spp)}
    \KeywordTok{diag}\NormalTok{(varcov) <-}\StringTok{ }\NormalTok{sigE}
    
    \CommentTok{# crank through nearPD to fix rounding errors}
    \NormalTok{if(sigE >}\StringTok{ }\DecValTok{0}\NormalTok{) \{ varcov <-}\StringTok{ }\NormalTok{Matrix::}\KeywordTok{nearPD}\NormalTok{(varcov)$mat \}  }
    
    \NormalTok{varcov <-}\StringTok{ }\KeywordTok{as.matrix}\NormalTok{(varcov)}
    \NormalTok{e      <-}\StringTok{ }\KeywordTok{rmvnorm}\NormalTok{(}\DataTypeTok{n =} \NormalTok{nTime, }\DataTypeTok{mean =} \KeywordTok{rep}\NormalTok{(}\DecValTok{0}\NormalTok{,num_spp), }\DataTypeTok{sigma =} \NormalTok{varcov)}
    \NormalTok{g      <-}\StringTok{ }\KeywordTok{exp}\NormalTok{(e) /}\StringTok{ }\NormalTok{(}\DecValTok{1}\NormalTok{+}\KeywordTok{exp}\NormalTok{(e))}
    \KeywordTok{return}\NormalTok{(g)}
  \NormalTok{\}}
  
  
  \NormalTok{####}
  \NormalTok{#### Simulate model -----------------------------------------------------}
  \NormalTok{####}
  \NormalTok{days           <-}\StringTok{ }\KeywordTok{c}\NormalTok{(}\DecValTok{1}\NormalTok{:days_to_track)}
  \NormalTok{num_spp        <-}\StringTok{ }\KeywordTok{length}\NormalTok{(parms$r)}
  \NormalTok{nmsDNR         <-}\StringTok{ }\KeywordTok{names}\NormalTok{(DNR)}
  \NormalTok{dormants       <-}\StringTok{ }\KeywordTok{grep}\NormalTok{(}\StringTok{"D"}\NormalTok{, }\KeywordTok{names}\NormalTok{(DNR))}
  \NormalTok{NR             <-}\StringTok{ }\NormalTok{DNR[-dormants] }
  \NormalTok{nmsNR          <-}\StringTok{ }\KeywordTok{names}\NormalTok{(NR)}
  \NormalTok{gVec           <-}\StringTok{ }\KeywordTok{getG}\NormalTok{(}\DataTypeTok{sigE =} \NormalTok{sigE, }\DataTypeTok{rho =} \NormalTok{rho, }\DataTypeTok{nTime =} \NormalTok{seasons, }\DataTypeTok{num_spp =} \NormalTok{num_spp)}
  \NormalTok{Rvector        <-}\StringTok{ }\KeywordTok{rlnorm}\NormalTok{(seasons, Rmu, Rsd_annual)}
  \NormalTok{saved_outs     <-}\StringTok{ }\KeywordTok{matrix}\NormalTok{(}\DataTypeTok{ncol=}\KeywordTok{length}\NormalTok{(DNR), }\DataTypeTok{nrow=}\NormalTok{seasons}\DecValTok{+1}\NormalTok{)}
  \NormalTok{saved_outs[}\DecValTok{1}\NormalTok{,] <-}\StringTok{ }\NormalTok{DNR }

  \NormalTok{##  Loop over seasons}
  \NormalTok{for(season_now in }\DecValTok{1}\NormalTok{:seasons) \{}
    \CommentTok{# Simulate continuous growing  season}
    \NormalTok{output   <-}\StringTok{ }\KeywordTok{ode}\NormalTok{(}\DataTypeTok{y =} \NormalTok{NR, }\DataTypeTok{times=}\NormalTok{days, }\DataTypeTok{func =} \NormalTok{updateNR, }\DataTypeTok{parms =} \NormalTok{parms)}
    \NormalTok{NR       <-}\StringTok{ }\NormalTok{output[}\KeywordTok{nrow}\NormalTok{(output),nmsNR]}
    \NormalTok{dormants <-}\StringTok{ }\KeywordTok{grep}\NormalTok{(}\StringTok{"D"}\NormalTok{, }\KeywordTok{names}\NormalTok{(DNR)) }
    \NormalTok{DNR      <-}\StringTok{ }\KeywordTok{c}\NormalTok{(DNR[dormants], NR)}
    
    \CommentTok{# Save end of season biomasses, before discrete transitions}
    \NormalTok{saved_outs[season_now}\DecValTok{+1}\NormalTok{,] <-}\StringTok{ }\NormalTok{DNR}
    
    \KeywordTok{names}\NormalTok{(DNR) <-}\StringTok{ }\NormalTok{nmsDNR}
    \NormalTok{DNR <-}\StringTok{ }\KeywordTok{update_DNR}\NormalTok{(season_now, DNR, gVec[season_now,],}
                      \DataTypeTok{alpha1 =} \NormalTok{alpha1, }\DataTypeTok{alpha2 =} \NormalTok{alpha2, }
                      \DataTypeTok{alpha3 =} \NormalTok{alpha3, }\DataTypeTok{alpha4 =} \NormalTok{alpha4,}
                      \DataTypeTok{eta1 =} \NormalTok{eta1, }\DataTypeTok{eta2 =} \NormalTok{eta2, }\DataTypeTok{eta3 =} \NormalTok{eta3, }\DataTypeTok{eta4 =} \NormalTok{eta4,}
                      \DataTypeTok{beta1 =} \NormalTok{beta1, }\DataTypeTok{beta2 =} \NormalTok{beta2, }
                      \DataTypeTok{beta3 =} \NormalTok{beta3, }\DataTypeTok{beta4 =} \NormalTok{beta4,}
                      \DataTypeTok{theta1 =} \NormalTok{theta1, }\DataTypeTok{theta2 =} \NormalTok{theta2, }
                      \DataTypeTok{theta3 =} \NormalTok{theta3, }\DataTypeTok{theta4 =} \NormalTok{theta4,}
                      \DataTypeTok{nu=}\NormalTok{nu)}
    
    \KeywordTok{names}\NormalTok{(DNR) <-}\StringTok{ }\NormalTok{nmsDNR}
    \NormalTok{NR         <-}\StringTok{ }\NormalTok{DNR[-dormants]  }
    \KeywordTok{names}\NormalTok{(NR)  <-}\StringTok{ }\NormalTok{nmsNR}
  \NormalTok{\} }\CommentTok{# next season}
  
  \KeywordTok{return}\NormalTok{(saved_outs)}
  
\NormalTok{\} ##  End simulation function}
\end{Highlighting}
\end{Shaded}

\newpage{}

\section{Additional Figures}

\begin{figure}[!ht]
  \centering
      \includegraphics[width=3in]{./components/fourspp_Ruptake_relnonlin.png}
  \caption{Resource uptake curves for each species (represented by different colors) as used in relative nonlinearity simulations. The equation for resource uptake is: $f_{i}(R) = r_{i}R^{a_{i}} / (b_{i}^{a_{i}}+R^{a_{i}})$. Parameter values are as follows. Species 1: \emph{r} = 0.2, \emph{a} = 2, \emph{b} = 2.5; Species 2: \emph{r} = 1, \emph{a} = 5, \emph{b} = 20; Species 3: \emph{r} = 2, \emph{a} = 10, \emph{b} = 30; Species 4: \emph{r} = 5, \emph{a} = 25, \emph{b} = 45.}
\end{figure}

\newpage{}

\begin{figure}[!ht]
  \centering
      \includegraphics[width=6.5in]{./components/SI_invasion_factorials.png}
  \caption{Variability of community biomass and invasion growth rates of the inferior competitor in a two-species community under different parameter combinations. Points are mean values from 10,000 growing seasons and lines are linear fits to show trends. In \textbf{Storage Effect} plots, resource supply is held constant between growing seasons, whereas resource supply varies each year in \textbf{Relative Nonlinearity} simulations.}
\end{figure}

\newpage{}

\begin{figure}[!ht]
  \centering
      \includegraphics[width=5in]{./components/storage_effect_div+envar_varycomp_loglog_slopes.png}
  \caption{Slopes of linear fits for the relationship between log(\emph{CV}) and log($\sigma_E$) at different levels of realized species richness from storage effect simulations. The slopes come from linear models fit to log-transformed versions of Figure 3 in the main text. For these simulations, ``symmetric competion'' (\tikzcircle{1.5pt}) refers to similar live-to-dormant biomass allocation fractions ($\boldsymbol{\alpha} = [0.5, 0.495, 0.49, 0.485]$ for the four species), and ``asymmetric competition'' (\tikzcircle[fill=blue]{1.5pt}) refers to more dissimilar live-to-dormant biomass allocation fractions ($\boldsymbol{\alpha} = [0.5, 0.49, 0.48, 0.47]$ for the four species).}
\end{figure}

\newpage{}

\begin{figure}[!ht]
  \centering
      \includegraphics[width=5in]{./components/relative_nonlinearity_div+envar_loglog_slopes.png}
  \caption{Slopes of linear fits for the relationship between log(\emph{CV}) and log($\sigma_R$) at different levels of realized species richness from relative nonlinearity simulations. The slopes come from linear models fit to log-transformed versions of Figure 4 in the main text.}
\end{figure}


\end{document}
