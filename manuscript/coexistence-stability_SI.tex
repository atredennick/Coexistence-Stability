\documentclass[11pt,]{article}
\usepackage{lmodern}
\usepackage{amssymb,amsmath}
\usepackage{ifxetex,ifluatex}
\usepackage{fixltx2e} % provides \textsubscript
\ifnum 0\ifxetex 1\fi\ifluatex 1\fi=0 % if pdftex
  \usepackage[T1]{fontenc}
  \usepackage[utf8]{inputenc}
\else % if luatex or xelatex
  \ifxetex
    \usepackage{mathspec}
  \else
    \usepackage{fontspec}
  \fi
  \defaultfontfeatures{Ligatures=TeX,Scale=MatchLowercase}
\fi
% use upquote if available, for straight quotes in verbatim environments
\IfFileExists{upquote.sty}{\usepackage{upquote}}{}
% use microtype if available
\IfFileExists{microtype.sty}{%
\usepackage{microtype}
\UseMicrotypeSet[protrusion]{basicmath} % disable protrusion for tt fonts
}{}
\usepackage[margin=1in]{geometry}
\usepackage{hyperref}
\PassOptionsToPackage{usenames,dvipsnames}{color} % color is loaded by hyperref
\hypersetup{unicode=true,
            colorlinks=true,
            linkcolor=black,
            citecolor=Blue,
            urlcolor=black,
            breaklinks=true}
\urlstyle{same}  % don't use monospace font for urls
\usepackage{color}
\usepackage{fancyvrb}
\newcommand{\VerbBar}{|}
\newcommand{\VERB}{\Verb[commandchars=\\\{\}]}
\DefineVerbatimEnvironment{Highlighting}{Verbatim}{commandchars=\\\{\}}
% Add ',fontsize=\small' for more characters per line
\usepackage{framed}
\definecolor{shadecolor}{RGB}{248,248,248}
\newenvironment{Shaded}{\begin{snugshade}}{\end{snugshade}}
\newcommand{\KeywordTok}[1]{\textcolor[rgb]{0.13,0.29,0.53}{\textbf{{#1}}}}
\newcommand{\DataTypeTok}[1]{\textcolor[rgb]{0.13,0.29,0.53}{{#1}}}
\newcommand{\DecValTok}[1]{\textcolor[rgb]{0.00,0.00,0.81}{{#1}}}
\newcommand{\BaseNTok}[1]{\textcolor[rgb]{0.00,0.00,0.81}{{#1}}}
\newcommand{\FloatTok}[1]{\textcolor[rgb]{0.00,0.00,0.81}{{#1}}}
\newcommand{\ConstantTok}[1]{\textcolor[rgb]{0.00,0.00,0.00}{{#1}}}
\newcommand{\CharTok}[1]{\textcolor[rgb]{0.31,0.60,0.02}{{#1}}}
\newcommand{\SpecialCharTok}[1]{\textcolor[rgb]{0.00,0.00,0.00}{{#1}}}
\newcommand{\StringTok}[1]{\textcolor[rgb]{0.31,0.60,0.02}{{#1}}}
\newcommand{\VerbatimStringTok}[1]{\textcolor[rgb]{0.31,0.60,0.02}{{#1}}}
\newcommand{\SpecialStringTok}[1]{\textcolor[rgb]{0.31,0.60,0.02}{{#1}}}
\newcommand{\ImportTok}[1]{{#1}}
\newcommand{\CommentTok}[1]{\textcolor[rgb]{0.56,0.35,0.01}{\textit{{#1}}}}
\newcommand{\DocumentationTok}[1]{\textcolor[rgb]{0.56,0.35,0.01}{\textbf{\textit{{#1}}}}}
\newcommand{\AnnotationTok}[1]{\textcolor[rgb]{0.56,0.35,0.01}{\textbf{\textit{{#1}}}}}
\newcommand{\CommentVarTok}[1]{\textcolor[rgb]{0.56,0.35,0.01}{\textbf{\textit{{#1}}}}}
\newcommand{\OtherTok}[1]{\textcolor[rgb]{0.56,0.35,0.01}{{#1}}}
\newcommand{\FunctionTok}[1]{\textcolor[rgb]{0.00,0.00,0.00}{{#1}}}
\newcommand{\VariableTok}[1]{\textcolor[rgb]{0.00,0.00,0.00}{{#1}}}
\newcommand{\ControlFlowTok}[1]{\textcolor[rgb]{0.13,0.29,0.53}{\textbf{{#1}}}}
\newcommand{\OperatorTok}[1]{\textcolor[rgb]{0.81,0.36,0.00}{\textbf{{#1}}}}
\newcommand{\BuiltInTok}[1]{{#1}}
\newcommand{\ExtensionTok}[1]{{#1}}
\newcommand{\PreprocessorTok}[1]{\textcolor[rgb]{0.56,0.35,0.01}{\textit{{#1}}}}
\newcommand{\AttributeTok}[1]{\textcolor[rgb]{0.77,0.63,0.00}{{#1}}}
\newcommand{\RegionMarkerTok}[1]{{#1}}
\newcommand{\InformationTok}[1]{\textcolor[rgb]{0.56,0.35,0.01}{\textbf{\textit{{#1}}}}}
\newcommand{\WarningTok}[1]{\textcolor[rgb]{0.56,0.35,0.01}{\textbf{\textit{{#1}}}}}
\newcommand{\AlertTok}[1]{\textcolor[rgb]{0.94,0.16,0.16}{{#1}}}
\newcommand{\ErrorTok}[1]{\textcolor[rgb]{0.64,0.00,0.00}{\textbf{{#1}}}}
\newcommand{\NormalTok}[1]{{#1}}
\usepackage{graphicx,grffile}
\makeatletter
\def\maxwidth{\ifdim\Gin@nat@width>\linewidth\linewidth\else\Gin@nat@width\fi}
\def\maxheight{\ifdim\Gin@nat@height>\textheight\textheight\else\Gin@nat@height\fi}
\makeatother
% Scale images if necessary, so that they will not overflow the page
% margins by default, and it is still possible to overwrite the defaults
% using explicit options in \includegraphics[width, height, ...]{}
\setkeys{Gin}{width=\maxwidth,height=\maxheight,keepaspectratio}
\IfFileExists{parskip.sty}{%
\usepackage{parskip}
}{% else
\setlength{\parindent}{0pt}
\setlength{\parskip}{6pt plus 2pt minus 1pt}
}
\setlength{\emergencystretch}{3em}  % prevent overfull lines
\providecommand{\tightlist}{%
  \setlength{\itemsep}{0pt}\setlength{\parskip}{0pt}}
\setcounter{secnumdepth}{5}
% Redefines (sub)paragraphs to behave more like sections
\ifx\paragraph\undefined\else
\let\oldparagraph\paragraph
\renewcommand{\paragraph}[1]{\oldparagraph{#1}\mbox{}}
\fi
\ifx\subparagraph\undefined\else
\let\oldsubparagraph\subparagraph
\renewcommand{\subparagraph}[1]{\oldsubparagraph{#1}\mbox{}}
\fi

%%% Use protect on footnotes to avoid problems with footnotes in titles
\let\rmarkdownfootnote\footnote%
\def\footnote{\protect\rmarkdownfootnote}

%%% Change title format to be more compact
\usepackage{titling}

% Create subtitle command for use in maketitle
\newcommand{\subtitle}[1]{
  \posttitle{
    \begin{center}\large#1\end{center}
    }
}

\setlength{\droptitle}{-2em}
  \title{}
  \pretitle{\vspace{\droptitle}}
  \posttitle{}
  \author{}
  \preauthor{}\postauthor{}
  \date{}
  \predate{}\postdate{}

\usepackage{lineno}
\linenumbers
\usepackage{setspace}
\usepackage{todonotes}
\onehalfspacing
\usepackage{rotating}
\usepackage{color, soul}
\usepackage[font={normalsize},labelfont={bf},labelsep=quad]{caption}
\usepackage{tikz}
\usepackage{bm,mathrsfs}
\usepackage{mathptmx}

\begin{document}

\newcommand{\tikzcircle}[2][red,fill=red]{\tikz[baseline=-0.5ex]\draw[#1,radius=#2] (0,0) circle ;}
\renewcommand\linenumberfont{\normalfont\tiny\sffamily\color{gray}}
\renewcommand\thefigure{S1-\arabic{figure}}  
\renewcommand\thetable{S1-\arabic{table}}  
\renewcommand\thesection{Section SI.\arabic{section}}

\begin{center}
\textbf{\Large{Supporting Information}} \\
A.T. Tredennick, P.B. Adler, \& F.R. Adler, ``The relationship between species richness and...'' \\
\emph{Ecology Letters}
\end{center}

\section{R Code for Consumer-Resource Model}

Below is the R code for our model function, which is represented
mathematically in the main text in Equations 1-4. The same code, along
with all the code to reproduce our results, has been archived on
Figshare (link) and is available on GitHub
(\url{http://github.com/atredennick/Coexistence-Stability/releases}).

\begin{Shaded}
\begin{Highlighting}[]
\NormalTok{simulate_model <-}\StringTok{ }\NormalTok{function(seasons, days_to_track, Rmu, }
                           \NormalTok{Rsd_annual, sigE, rho, }
                           \NormalTok{alpha1, alpha2, alpha3, alpha4,}
                           \NormalTok{eta1, eta2, eta3, eta4,}
                           \NormalTok{r1, r2, r3, r4,}
                           \NormalTok{a1, a2, a3, a4,}
                           \NormalTok{b1, b2, b3, b4,}
                           \NormalTok{eps1, eps2, eps3, eps4,}
                           \NormalTok{D1, D2, D3, D4,}
                           \NormalTok{N1, N2, N3, N4, R) \{}
  
  \KeywordTok{require}\NormalTok{(}\StringTok{'deSolve'}\NormalTok{) }\CommentTok{# for solving continuous differential equations}
  \KeywordTok{require}\NormalTok{(}\StringTok{'mvtnorm'}\NormalTok{) }\CommentTok{# for multivariate normal distribution functions}
  
  \NormalTok{##  Assign parameter values to appropriate lists}
  \NormalTok{DNR <-}\StringTok{ }\KeywordTok{c}\NormalTok{(}\DataTypeTok{D=}\KeywordTok{c}\NormalTok{(D1,D2,D3,D4),   }\CommentTok{# initial dormant state abundance}
           \DataTypeTok{N=}\KeywordTok{c}\NormalTok{(N1,N2,N3,N4),   }\CommentTok{# initial live state abundance}
           \DataTypeTok{R=}\NormalTok{R)                }\CommentTok{# initial resource level}
  
  \NormalTok{parms <-}\StringTok{ }\KeywordTok{list} \NormalTok{(}
    \DataTypeTok{r   =} \KeywordTok{c}\NormalTok{(r1,r2,r3,r4),          }\CommentTok{# max growth rate for each species}
    \DataTypeTok{a   =} \KeywordTok{c}\NormalTok{(a1,a2,a3,a4),          }\CommentTok{# rate parameter for Hill function }
    \DataTypeTok{b   =} \KeywordTok{c}\NormalTok{(b1,b2,b3,b4),          }\CommentTok{# shape parameter for Hill function}
    \DataTypeTok{eps =} \KeywordTok{c}\NormalTok{(eps1,eps2,eps3,eps4)   }\CommentTok{# resource-to-biomass efficiency}
  \NormalTok{)}
  
  
  \NormalTok{####}
  \NormalTok{####  Sub-Model functions ----------------------------------------------------}
  \NormalTok{####}
  \NormalTok{## Continuous model}
  \NormalTok{updateNR <-}\StringTok{ }\NormalTok{function(t, NR, parms)\{}
    \KeywordTok{with}\NormalTok{(}\KeywordTok{as.list}\NormalTok{(}\KeywordTok{c}\NormalTok{(NR, parms)), \{}
      \NormalTok{dN1dt =}\StringTok{ }\NormalTok{N1*eps[}\DecValTok{1}\NormalTok{]*(}\KeywordTok{uptake_R}\NormalTok{(r[}\DecValTok{1}\NormalTok{], R, a[}\DecValTok{1}\NormalTok{], b[}\DecValTok{1}\NormalTok{]))}
      \NormalTok{dN2dt =}\StringTok{ }\NormalTok{N2*eps[}\DecValTok{2}\NormalTok{]*(}\KeywordTok{uptake_R}\NormalTok{(r[}\DecValTok{2}\NormalTok{], R, a[}\DecValTok{2}\NormalTok{], b[}\DecValTok{2}\NormalTok{]))}
      \NormalTok{dN3dt =}\StringTok{ }\NormalTok{N3*eps[}\DecValTok{3}\NormalTok{]*(}\KeywordTok{uptake_R}\NormalTok{(r[}\DecValTok{3}\NormalTok{], R, a[}\DecValTok{3}\NormalTok{], b[}\DecValTok{3}\NormalTok{]))}
      \NormalTok{dN4dt =}\StringTok{ }\NormalTok{N4*eps[}\DecValTok{4}\NormalTok{]*(}\KeywordTok{uptake_R}\NormalTok{(r[}\DecValTok{4}\NormalTok{], R, a[}\DecValTok{4}\NormalTok{], b[}\DecValTok{4}\NormalTok{]))}
      \NormalTok{dRdt  =}\StringTok{ }\NormalTok{-}\DecValTok{1} \NormalTok{*}\StringTok{ }\NormalTok{(dN1dt/eps[}\DecValTok{1}\NormalTok{] +}\StringTok{ }\NormalTok{dN2dt/eps[}\DecValTok{2}\NormalTok{] +}\StringTok{ }\NormalTok{dN3dt/eps[}\DecValTok{3}\NormalTok{] +}\StringTok{ }\NormalTok{dN4dt/eps[}\DecValTok{4}\NormalTok{])}
      \KeywordTok{list}\NormalTok{(}\KeywordTok{c}\NormalTok{(dN1dt, dN2dt, dN3dt, dN4dt, dRdt)) }\CommentTok{# output as list}
    \NormalTok{\})}
  \NormalTok{\} }\CommentTok{# end continuous function}
  
  \NormalTok{## Discrete model}
  \NormalTok{update_DNR <-}\StringTok{ }\NormalTok{function(t, DNR, gammas,}
                         \NormalTok{alpha1, alpha2, alpha3, alpha4,}
                         \NormalTok{eta1, eta2, eta3, eta4) \{}
    \KeywordTok{with} \NormalTok{(}\KeywordTok{as.list}\NormalTok{(DNR),\{}
      \NormalTok{g1    <-}\StringTok{ }\NormalTok{gammas[}\DecValTok{1}\NormalTok{]}
      \NormalTok{g2    <-}\StringTok{ }\NormalTok{gammas[}\DecValTok{2}\NormalTok{]}
      \NormalTok{g3    <-}\StringTok{ }\NormalTok{gammas[}\DecValTok{3}\NormalTok{]}
      \NormalTok{g4    <-}\StringTok{ }\NormalTok{gammas[}\DecValTok{4}\NormalTok{]}
      \NormalTok{D1new <-}\StringTok{ }\NormalTok{(}\DecValTok{1}\NormalTok{-g1)*(alpha1*N1 +}\StringTok{ }\NormalTok{D1)*(}\DecValTok{1}\NormalTok{-eta1)}
      \NormalTok{D2new <-}\StringTok{ }\NormalTok{(}\DecValTok{1}\NormalTok{-g2)*(alpha2*N2 +}\StringTok{ }\NormalTok{D2)*(}\DecValTok{1}\NormalTok{-eta2)}
      \NormalTok{D3new <-}\StringTok{ }\NormalTok{(}\DecValTok{1}\NormalTok{-g3)*(alpha3*N3 +}\StringTok{ }\NormalTok{D3)*(}\DecValTok{1}\NormalTok{-eta3)}
      \NormalTok{D4new <-}\StringTok{ }\NormalTok{(}\DecValTok{1}\NormalTok{-g4)*(alpha4*N4 +}\StringTok{ }\NormalTok{D4)*(}\DecValTok{1}\NormalTok{-eta4)}
      \NormalTok{N1new <-}\StringTok{ }\NormalTok{g1*(alpha1*N1 +}\StringTok{ }\NormalTok{D1)*(}\DecValTok{1}\NormalTok{-eta1)}
      \NormalTok{N2new <-}\StringTok{ }\NormalTok{g2*(alpha2*N2 +}\StringTok{ }\NormalTok{D2)*(}\DecValTok{1}\NormalTok{-eta2)}
      \NormalTok{N3new <-}\StringTok{ }\NormalTok{g3*(alpha3*N3 +}\StringTok{ }\NormalTok{D3)*(}\DecValTok{1}\NormalTok{-eta3)}
      \NormalTok{N4new <-}\StringTok{ }\NormalTok{g4*(alpha4*N4 +}\StringTok{ }\NormalTok{D4)*(}\DecValTok{1}\NormalTok{-eta4)}
      \NormalTok{Rnew  <-}\StringTok{ }\NormalTok{Rvector[t]}
      \KeywordTok{return}\NormalTok{(}\KeywordTok{c}\NormalTok{(D1new, D2new, D3new, D4new, N1new, N2new, N3new, N4new, Rnew))}
    \NormalTok{\})}
  \NormalTok{\}}
  
  \NormalTok{##  Resource uptake function (Hill function)}
  \NormalTok{uptake_R <-}\StringTok{ }\NormalTok{function(r, R, a, b) \{}
    \KeywordTok{return}\NormalTok{((r*R^a) /}\StringTok{ }\NormalTok{(b^a +}\StringTok{ }\NormalTok{R^a))}
  \NormalTok{\}}
  
  \NormalTok{##  Generate germination fractions}
  \NormalTok{getG <-}\StringTok{ }\NormalTok{function(sigE, rho, nTime, num_spp) \{}
    \NormalTok{varcov       <-}\StringTok{ }\KeywordTok{matrix}\NormalTok{(}\KeywordTok{rep}\NormalTok{(rho*sigE,num_spp*}\DecValTok{2}\NormalTok{), num_spp, num_spp)}
    \KeywordTok{diag}\NormalTok{(varcov) <-}\StringTok{ }\NormalTok{sigE}
    \NormalTok{if(sigE >}\StringTok{ }\DecValTok{0}\NormalTok{) \{ varcov <-}\StringTok{ }\NormalTok{Matrix::}\KeywordTok{nearPD}\NormalTok{(varcov)$mat \} }\CommentTok{# crank through nearPD to fix rounding errors }
    \NormalTok{varcov <-}\StringTok{ }\KeywordTok{as.matrix}\NormalTok{(varcov)}
    \NormalTok{e      <-}\StringTok{ }\KeywordTok{rmvnorm}\NormalTok{(}\DataTypeTok{n =} \NormalTok{nTime, }\DataTypeTok{mean =} \KeywordTok{rep}\NormalTok{(}\DecValTok{0}\NormalTok{,num_spp), }\DataTypeTok{sigma =} \NormalTok{varcov)}
    \NormalTok{g      <-}\StringTok{ }\KeywordTok{exp}\NormalTok{(e) /}\StringTok{ }\NormalTok{(}\DecValTok{1}\NormalTok{+}\KeywordTok{exp}\NormalTok{(e))}
    \KeywordTok{return}\NormalTok{(g)}
  \NormalTok{\}}
  
  
  \NormalTok{####}
  \NormalTok{#### Simulate model -----------------------------------------------------}
  \NormalTok{####}
  \NormalTok{days           <-}\StringTok{ }\KeywordTok{c}\NormalTok{(}\DecValTok{1}\NormalTok{:days_to_track)}
  \NormalTok{num_spp        <-}\StringTok{ }\KeywordTok{length}\NormalTok{(parms$r)}
  \NormalTok{nmsDNR         <-}\StringTok{ }\KeywordTok{names}\NormalTok{(DNR)}
  \NormalTok{dormants       <-}\StringTok{ }\KeywordTok{grep}\NormalTok{(}\StringTok{"D"}\NormalTok{, }\KeywordTok{names}\NormalTok{(DNR))}
  \NormalTok{NR             <-}\StringTok{ }\NormalTok{DNR[-dormants] }
  \NormalTok{nmsNR          <-}\StringTok{ }\KeywordTok{names}\NormalTok{(NR)}
  \NormalTok{gVec           <-}\StringTok{ }\KeywordTok{getG}\NormalTok{(}\DataTypeTok{sigE =} \NormalTok{sigE, }\DataTypeTok{rho =} \NormalTok{rho, }\DataTypeTok{nTime =} \NormalTok{seasons, }\DataTypeTok{num_spp =} \NormalTok{num_spp)}
  \NormalTok{Rvector        <-}\StringTok{ }\KeywordTok{rlnorm}\NormalTok{(seasons, Rmu, Rsd_annual)}
  \NormalTok{saved_outs     <-}\StringTok{ }\KeywordTok{matrix}\NormalTok{(}\DataTypeTok{ncol=}\KeywordTok{length}\NormalTok{(DNR), }\DataTypeTok{nrow=}\NormalTok{seasons}\DecValTok{+1}\NormalTok{)}
  \NormalTok{saved_outs[}\DecValTok{1}\NormalTok{,] <-}\StringTok{ }\NormalTok{DNR }

  \NormalTok{##  Loop over seasons}
  \NormalTok{for(season_now in }\DecValTok{1}\NormalTok{:seasons) \{}
    \CommentTok{# Simulate continuous growing  season}
    \NormalTok{output   <-}\StringTok{ }\KeywordTok{ode}\NormalTok{(}\DataTypeTok{y =} \NormalTok{NR, }\DataTypeTok{times=}\NormalTok{days, }\DataTypeTok{func =} \NormalTok{updateNR, }\DataTypeTok{parms =} \NormalTok{parms)}
    \NormalTok{NR       <-}\StringTok{ }\NormalTok{output[}\KeywordTok{nrow}\NormalTok{(output),nmsNR]}
    \NormalTok{dormants <-}\StringTok{ }\KeywordTok{grep}\NormalTok{(}\StringTok{"D"}\NormalTok{, }\KeywordTok{names}\NormalTok{(DNR)) }
    \NormalTok{DNR      <-}\StringTok{ }\KeywordTok{c}\NormalTok{(DNR[dormants], NR)}
    
    \CommentTok{# Save end of season biomasses, before discrete transitions}
    \NormalTok{saved_outs[season_now}\DecValTok{+1}\NormalTok{,] <-}\StringTok{ }\NormalTok{DNR}
    
    \KeywordTok{names}\NormalTok{(DNR) <-}\StringTok{ }\NormalTok{nmsDNR}
    \NormalTok{DNR        <-}\StringTok{ }\KeywordTok{update_DNR}\NormalTok{(season_now, DNR, gVec[season_now,],}
                             \DataTypeTok{alpha1 =} \NormalTok{alpha1, }\DataTypeTok{alpha2 =} \NormalTok{alpha2, }
                             \DataTypeTok{alpha3 =} \NormalTok{alpha3, }\DataTypeTok{alpha4 =} \NormalTok{alpha4,}
                             \DataTypeTok{eta1 =} \NormalTok{eta1, }\DataTypeTok{eta2 =} \NormalTok{eta2, }\DataTypeTok{eta3 =} \NormalTok{eta3, }\DataTypeTok{eta4 =} \NormalTok{eta4)}
    
    \KeywordTok{names}\NormalTok{(DNR) <-}\StringTok{ }\NormalTok{nmsDNR}
    \NormalTok{NR         <-}\StringTok{ }\NormalTok{DNR[-dormants]  }
    \KeywordTok{names}\NormalTok{(NR)  <-}\StringTok{ }\NormalTok{nmsNR}
  \NormalTok{\} }\CommentTok{# next season}
  
  \KeywordTok{return}\NormalTok{(saved_outs)}
  
\NormalTok{\} }\CommentTok{#end simulation function}
\end{Highlighting}
\end{Shaded}

\section{Exploring Parameter Space}

In the main text we presented results that most clearly demonstrated our
main point, sacrificing some quantitative rigor in terms of exploring
the parameter space of the model. In many cases, altering model
parameters, or making them asymmetric among species, makes coexistence
more difficult. For example, as we show in Figure 3 in the main text,
imposing competitive hierarchies makes it more difficult for species to
coexist, but it does not impact our conclusion that a positive
diversity--ecosystem variability relationship is possible. We imposed
competitive hierarchies by making the live-to-dormant biomass fractions
unique for each species, but any parameter that controls population
growth would do the same. For example, if we make dormant mortality
rates assymetric among species we achieve similar results as shown in
Figure 3 (Fig. S1-1).

\begin{figure}[!ht]
  \centering
      \includegraphics[width=6in]{./components/SI_storage_effect_asymmetric_etas.png}
  \caption{The effect of increasing environmental variability on ecosystem variability when species coexist via the storage effect. Top panels show simulation results where species have relatively symmetric dormant mortality rates ($\eta_1=0.1, \eta_2=0.115, \eta_3=0.12, \eta_4=0.125$), whereas bottom panels show results with slightly more asymmetric dormant mortality rates ($\eta_1=0.1, \eta_2=0.12, \eta_3=0.13, \eta_4=0.14$). Columns show results for different levels of correlations of species' environmental responses, $\rho = -1/3$ and $\rho = 0$. Colored vertical lines show the magnitude of environmental variability at which each level of species richness first occurs. Parameter values are as in Figure 2A except for $\alpha$s: $\alpha_1 = \alpha_2 = \alpha_3 = \alpha_4 = 0.5$.}
\end{figure}

While the results of competitive assymetries among species are rather
intuitive, the sensitivity of our results to changes in the absolute
values of parameters is less intuitive and requires exploration. We took
a targeted approach to exploring parameter space by focusing on
particular processes. First, we examined the sensitivity of our results
to parameters that control the strength of each coexistence mechanism.
Second, we examined the sensitivity of our results to parameters that
previous theory have identified as important for the
diversity--ecosystem variability relationship. We describe each in turn.

\subsection{Parameter sensitivity: coexistence strength}

For the storage effect, coexistence strength declines as:

\begin{enumerate}
\def\labelenumi{\arabic{enumi}.}
\tightlist
\item
  Environment-competition covariance (\emph{EC} covariance) becomes less
  positive
\item
  Environemntal variation decreases
\item
  Buffering of population growth rate declines
\end{enumerate}

Conditions 1 \& 2 are already included in our main analysis because we
present results across a spectrum of correlations among species'
environmental responses (condition 1) and across gradients of
environmental variation (condition 2). Here, we show how the patterns
described in the main text change as the buffering of populaion growth
declines (condition 3). We do this by conducting the same simulations as
in Figure 3 of the main text, but for one set of live-to-dormant
transition rates (\(\alpha\)s) and with two levels (high and low) of
dormant mortality rates (\(\eta\)s).

Our results are qualitatively similar: (1) increasing environmental
variation allows more species to coexist, creating a positive
relationship between species richness and ecosystem variability; (2) and
at any given level of environmental variability, it is always better to
have more species, creating a within-site negative relationship between
species richness and ecosystem variability (Fig. S1-2). Although the
qualitative patterns are similar, increasing the dormant mortality rate
has two interesting effects: (1) it makes coexistence more difficult,
(2) it reduces the absolute value of ecosystem \emph{CV} by reducing
mean population size a little bit and by reducing the temporal standard
deviation a lot, and (3) it weakens the buffering effect of additional
species (compare the spread of the lines between top and bottom panels
of Fig. S1-2). The second result occurs because population fluctuations
are reduced when very little biomass can be activated to the live stage
at the beginning of each season. That is, in terms of an annual plant,
germination of live biomass is always a fraction of a very small number.
The third result is a consequence of the second: populations are not
fluctuating that much, so total \emph{CV} can only be reduced by so much
with the addition of another species.

\begin{figure}[!ht]
  \centering
      \includegraphics[width=6in]{./components/SI_storage_effect_two_etas.png}
  \caption{The effect of increasing environmental variability on ecosystem variability when species coexist via the storage effect. Top panels show simulation results where species have low dormant mortality rates ($\eta = 0.2$), whereas bottom panels show results with high dormant mortality rates ($\eta = 0.8$). Columns show results for different levels of correlations of species' environmental responses, $\rho = -1/3$ and $\rho = 0$. Colored vertical lines show the magnitude of environmental variability at which each level of species richness first occurs. Parameter values are as in Figure 2A except for $\alpha$s: $\alpha_1 = 0.5, \alpha_2 = 0.495, \alpha_3 = 0.49, \alpha_4 = 0.485$.}
\end{figure}

For relative nonlinearity, coexistence strength declines as:

\begin{enumerate}
\def\labelenumi{\arabic{enumi}.}
\tightlist
\item
  Variability in population densities, here driven by resource
  variability, declines
\item
  Species' resource uptake curves become more similar
\item
  Mean resource level declines
\end{enumerate}

Condition 1 is already included in our main analysis (Figures 2 and 4).
Condition 2 simply creates conditions where some species cannot coexist,
which would not change the relationship between species richness and
ecosystem variability, but rather disallow it. Condition 3 is more
interesting, because it could create situations where species may not
coexist, but it could also weaken the diversity--ecosystem variabilty
relationship be reducing mean biomass and the effect of species with
very nonlinear growth rates. To explore the effect of condition 3, we
ran simulations across a gradient of resource variability (11 levels)
crossed with a gradient of mean resource levels (3 levels).

\subsection{Parameter sensitivity: diversity--ecosystem variability parameters}

Previous theory (Loreau \& de Mazancourt 2013) identifies three main
mechanisms by which diversity can reduce ecosystem variability:

\begin{enumerate}
\def\labelenumi{\arabic{enumi}.}
\tightlist
\item
  Aysnchrony of species' responses to environmental conditions
\item
  Reduced mean competition at the community-level
\item
  Differences in population growth rates, which differentiates the speed
  at which species respond to perturbations.
\end{enumerate}

\emph{CV} scaled by the mean, and we don't have demographic
stochasticity because we are assuming large populations, so reducing
overall competition to get over-yielding doesn't matter here\ldots{}

\section{Eight-Species Storage Effect Model}

In the main text we constrained our focus to four-species communities
because getting more than four species to coexist by relative
nonlinearity is tricky, and usually requires adding another coexistence
mechanism on top off relative nonlinearity (Yuan \& Chesson 2015). The
storage effect does not suffer from this limitation, but we wanted our
results in the main text to be easily comparable between coexistence
mechanisms. Here, we show that our results are qualitatively similar if
we simulate an eight-species community with species coexistence
maintained by the storage effect. We conducted the same numerical
simulations described in the main text for Figure 2. Quoting from the
main text:

\begin{quote}
\emph{First, we allowed the variance of the environment to determine how many species can coexist, akin to a community assembly experiment with a species pool of four species.
To do this, we simulated communities with all species initially present across a gradient of annual resource variability (for relative nonlinearity) or environmental cue variability (for the storage effect).
Second, we chose parameter values that allowed coexistence of all four species and then performed species removals, akin to a biodiversity--ecosystem function experiment.
The two simulation experiments correspond to (i) sampling ecosystem function across a natural gradient of species richness and (ii) sampling ecosystem function across diversity treatments within a site.}
\end{quote}

From one to four species, the relationship is as presented in the main
text: total community \emph{CV} increases approximately linearly with
environmental variability because (1) environmental variability promotes
species coexistence \emph{and} (2) environmental variability causes
populations fluctuations to increase (Figure S1-x). However, after four
species, the relationship saturates --- species additions due to
coexistence by the storage effect completely buffer ecosystem
variability from further increases in environmental variability (Figure
S1-x). Thus, our results provide novel theoretical explanations for
positive and flat diversity--ecosystem variability relationships.

\begin{figure}[!ht]
  \centering
      \includegraphics[width=4in]{./components/regional_diversity_stability_storage_effect_8species.png}
  \caption{Variability of total community biomass as a function of species richness when coexistence is maintained by the storage effect. Results are from simulations where environmental variance determines the number species that coexist in a community (e.g., a ``regional'' relationship. Colored points show results from individual simulations and the gray points with connecting line show the mean at each level of environmental variance.}
\end{figure}

\newpage{}

\section{Additional Figures}

\begin{figure}[!ht]
  \centering
      \includegraphics[width=3in]{./components/fourspp_Ruptake_relnonlin.png}
  \caption{Resource uptake curves for each species (represented by different colors) as used in relative nonlinearity simulations. The equation for resource uptake is: $f_{i}(R) = r_{i}R^{a_{i}} / (b_{i}^{a_{i}}+R^{a_{i}})$. Parameter values are as follows. Species 1: \emph{r} = 0.2, \emph{a} = 2, \emph{b} = 2.5; Species 2: \emph{r} = 1, \emph{a} = 5, \emph{b} = 20; Species 3: \emph{r} = 2, \emph{a} = 10, \emph{b} = 30; Species 4: \emph{r} = 5, \emph{a} = 25, \emph{b} = 45.}
\end{figure}

\newpage{}

\begin{figure}[!ht]
  \centering
      \includegraphics[width=4in]{./components/SI_invasion_factorials.png}
  \caption{Variability of community biomass and invasion growth rates of the inferior competitor in a two-species community under different parameter combinations. Points are mean values from 5,000 growing seasons and lines are linear fits to show trends. In \textbf{Storage Effect} plots (a,b), resource supply is held constant between growing seasons. Resource supply varies each year in \textbf{Relative Nonlinearity} simulations (c,d), while the environmental cue variance is set to 0.}
\end{figure}

\newpage{}

\begin{figure}[!ht]
  \centering
      \includegraphics[width=5in]{./components/storage_effect_div+envar_varycomp_loglog_slopes.png}
  \caption{Slopes of linear fits for the relationship between log(\emph{CV}) and log($\sigma_E$) at different levels of realized species richness from storage effect simulations. The slopes come from linear models fit to log-transformed versions of Figure 3 in the main text. For these simulations, ``symmetric competion'' (\tikzcircle{1.5pt}) refers to similar live-to-dormant biomass allocation fractions ($\boldsymbol{\alpha} = [0.5, 0.495, 0.49, 0.485]$ for the four species), and ``asymmetric competition'' (\tikzcircle[fill=blue]{1.5pt}) refers to more dissimilar live-to-dormant biomass allocation fractions ($\boldsymbol{\alpha} = [0.5, 0.49, 0.48, 0.47]$ for the four species).}
\end{figure}

\newpage{}

\begin{figure}[!ht]
  \centering
      \includegraphics[width=5in]{./components/relative_nonlinearity_div+envar_loglog_slopes.png}
  \caption{Slopes of linear fits for the relationship between log(\emph{CV}) and log($\sigma_R$) at different levels of realized species richness from relative nonlinearity simulations. The slopes come from linear models fit to log-transformed versions of Figure 4 in the main text.}
\end{figure}

\newpage{}

\begin{figure}[!ht]
  \centering
      \includegraphics[width=5in]{./components/SI_storage_effect_two_rmus_fourSpeciesOnly.png}
  \caption{Ecosystem \emph{CV} of a four-species community as a function of environmental cue variance at two levels of mean resource availability (\texttt{Rmu} across tops of panels) and two levels of correlation of species' responses to environmental conditions ($\rho$). Here we simualted community dynamics across a greater range of environmental cue variance (up to 10) to show that the \emph{CV} of four-species communities does not remain flat as environmental variance increases beyond levels shown in the main text. The mean resource level has no effect on the results.}
\end{figure}

\newpage{}

\section*{References}\label{references}
\addcontentsline{toc}{section}{References}

\hypertarget{refs}{}
\hypertarget{ref-Loreau2013}{}
Loreau, M. \& de Mazancourt, C. (2013). Biodiversity and ecosystem
stability: A synthesis of underlying mechanisms. \emph{Ecology Letters},
16, 106--115.

\hypertarget{ref-Yuan2015}{}
Yuan, C. \& Chesson, P. (2015). The relative importance of relative
nonlinearity and the storage effect in the lottery model.
\emph{Theoretical Population Biology}, 105, 39--52.


\end{document}
