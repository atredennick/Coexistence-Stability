\documentclass[12pt,]{article}
\usepackage{lmodern}
\usepackage{amssymb,amsmath}
\usepackage{ifxetex,ifluatex}
\usepackage{fixltx2e} % provides \textsubscript
\ifnum 0\ifxetex 1\fi\ifluatex 1\fi=0 % if pdftex
  \usepackage[T1]{fontenc}
  \usepackage[utf8]{inputenc}
\else % if luatex or xelatex
  \ifxetex
    \usepackage{mathspec}
    \usepackage{xltxtra,xunicode}
  \else
    \usepackage{fontspec}
  \fi
  \defaultfontfeatures{Mapping=tex-text,Scale=MatchLowercase}
  \newcommand{\euro}{€}
\fi
% use upquote if available, for straight quotes in verbatim environments
\IfFileExists{upquote.sty}{\usepackage{upquote}}{}
% use microtype if available
\IfFileExists{microtype.sty}{%
\usepackage{microtype}
\UseMicrotypeSet[protrusion]{basicmath} % disable protrusion for tt fonts
}{}
\usepackage[margin=1in]{geometry}
\usepackage{graphicx}
\makeatletter
\def\maxwidth{\ifdim\Gin@nat@width>\linewidth\linewidth\else\Gin@nat@width\fi}
\def\maxheight{\ifdim\Gin@nat@height>\textheight\textheight\else\Gin@nat@height\fi}
\makeatother
% Scale images if necessary, so that they will not overflow the page
% margins by default, and it is still possible to overwrite the defaults
% using explicit options in \includegraphics[width, height, ...]{}
\setkeys{Gin}{width=\maxwidth,height=\maxheight,keepaspectratio}
\ifxetex
  \usepackage[setpagesize=false, % page size defined by xetex
              unicode=false, % unicode breaks when used with xetex
              xetex]{hyperref}
\else
  \usepackage[unicode=true]{hyperref}
\fi
\hypersetup{breaklinks=true,
            bookmarks=true,
            pdfauthor={},
            pdftitle={},
            colorlinks=true,
            citecolor=blue,
            urlcolor=black,
            linkcolor=black,
            pdfborder={0 0 0}}
\urlstyle{same}  % don't use monospace font for urls
\setlength{\parindent}{0pt}
\setlength{\parskip}{6pt plus 2pt minus 1pt}
\setlength{\emergencystretch}{3em}  % prevent overfull lines
\setcounter{secnumdepth}{0}

%%% Use protect on footnotes to avoid problems with footnotes in titles
\let\rmarkdownfootnote\footnote%
\def\footnote{\protect\rmarkdownfootnote}

%%% Change title format to be more compact
\usepackage{titling}

% Create subtitle command for use in maketitle
\newcommand{\subtitle}[1]{
  \posttitle{
    \begin{center}\large#1\end{center}
    }
}

\setlength{\droptitle}{-2em}
  \title{}
  \pretitle{\vspace{\droptitle}}
  \posttitle{}
  \author{}
  \preauthor{}\postauthor{}
  \date{}
  \predate{}\postdate{}

\usepackage{lineno}
\linenumbers
\usepackage{setspace}
\usepackage{todonotes}
\doublespacing
\usepackage{rotating}
\usepackage{color, soul}
\usepackage[font={footnotesize},labelfont={sf,bf},labelsep=space]{caption}
\usepackage{sectsty}


\begin{document}

\maketitle


\renewcommand\linenumberfont{\normalfont\tiny\sffamily\color{gray}}

\allsectionsfont{\normalfont\sffamily\bfseries}

\begin{singlespace}

\begin{centering}

\textsf{\large{\textbf{How fluctuation-dependent species coexistence affects the diversity-stability relationship}}}



\vspace{2.5em}

\renewcommand*{\thefootnote}{\fnsymbol{footnote}}

Andrew T. Tredennick\textsuperscript{1}, Peter B. Adler\textsuperscript{1}, and Frederick R. Adler\textsuperscript{2}

\vspace{1.5em}

\textit{\small{\textsuperscript{1}Department of Wildland Resources and the Ecology Center, Utah State University, Logan, Utah 84322}} \\
\textit{\small{\textsuperscript{2}Departments of Biology and Mathematics, University of Utah, Salt Lake City, Utah}} 

\end{centering}

\vspace{3em}

\noindent \textbf{Keywords}: coexistence, storage effect, relative nonlinearity, diversity-stability hypothesis, pulsed differential equation, consumer-resource dynamics

\noindent \textbf{Authorship}: All authors conceived the research and designed the modeling approach; ATT conducted model simulations, with input from PBA and FRA; ATT wrote the manuscript and all authors contributed to revisions.

\noindent \textbf{Running Title}: Environmental variability, ecosystem stability, \& species coexistence

\noindent \textbf{Article Type}: Letter

\noindent \textbf{Number of Words}:

\noindent \textbf{Number of References}:

\noindent \textbf{Number of Tables and Figures}:

\noindent \textbf{Corresponding Author}:  \\
Andrew Tredennick  \\
Department of Wildland Resources and the Ecology Center  \\
Utah State University  \\
5230 Old Main Hill  \\
Logan, Utah 84322 USA  \\
Phone: +1-970-443-1599  \\
Fax: +1-435-797-3796  \\
Email: atredenn@gmail.com

\end{singlespace}

\newpage{}

\begin{abstract}
Theory relating species richness to ecosystem stability typically ignores interactions between environmental variability and species coexistence, or fluctuation-dependent coexistence.
This may explain why we lack general explanations for observed deviations from the predicted positive diversity-stability relationship.
It also limits our ability to predict the consequences of increasing environmental variability.
We use a consumer-resource model to explore how fluctuation-dependent coexistence via the storage effect and relative nonlinearity affects ecosystem stability.
We show that a negative, rather than positive, diversity-stability relationship is possible when ecosystem function is sampled across a natural diversity gradient.
We also describe how fluctuation-dependent coexistence can buffer ecosystem functioning against increasing environmental variability by allowing more species to coexist and thus contribute to portfolio effects.
Our work provides a general explanation for non-positive diversity-stability relalationships and highlights the importance of conserving regional species pools so that species can be added to fluctuation-dependent communities as environmental variability increases.
\vspace{2em}
\end{abstract}

\newpage{}

\setlength{\parindent}{5ex}

\subsection{INTRODUCTION}\label{introduction}

MacArthur (1955), Elton (1958), and even Darwin (Turnbull et al. 2013)
recognized that species can compensate for each other and stabilize
functioning in ecosystems subject to temporal variation in environmental
conditions. This idea underlies the ``insurance hypothesis'' (Yachi and
Loreau 1999), which suggests stability increases with diversity because
species respond dissimilarly to environmental conditions -- species A
has highest growth rates under conditions X whereas species B has
highest growth rates under conditions Y. More species confer temporal
stability by broadening the range of conditions under which the
community maintains function (Loreau 2010). Diverse models all predict a
positive relationship between species richness and ecosystem stability
(Lehman and Tilman 2000, Ives and Hughes 2002, Loreau and {{de
Mazancourt}} 2013), and experimental tests tend to support such a
prediction (Tilman et al. 2006, Hector et al. 2010). However, empirical
support for a positive diversity-stability relationship is not
unequivocal (Jiang and Pu 2009). Furthermore, theory on the relationship
between biodiversity and ecosystem stability often ignores the processes
that determine species coexistence in variable environments (Loreau
2010, but see Chesson et al. 2001).

Temporally fluctuating environmental conditions are an important
ingredient for stable species coexistence, both in theoretical models
(Chesson 2000, Chesson et al. 2004) and in natural communities
(C{á}ceres 1997, Descamps-Julien and Gonzalez 2005, Adler et al. 2006,
Angert et al. 2009). Such ``fluctuation-dependent'' coexistence (Chesson
2000) requires that species have unique environmental responses and that
environmental conditions vary enough for each coexisting species to
experience good and bad conditions. Thus, there is reason to expect
environmental variability to promote species richness when coexistence
is maintained by a fluctuation-dependent mechanism (Adler and Drake
2008). Of course, increasing environmental variability may also decrease
ecosystem stability through time by increasing the fluctuations of
individual species, regardless of species richness.

The countervailing effects of environmental variability present an
interesting paradox: increasing variability should decrease ecosystem
stability, but may also increase richness, which may offset the decrease
in stability. Such a paradox complicates predictions about how
ecosystems will respond as environmental conditions exceed historical
ranges of variability. The unknown net effect of environment variability
may be reflected in the mixed results from empirical studies on the
diversity-stability relationship. Observational tests of the
diversity-stability relationship, which require sampling across natural
diversity gradients, have yielded positive (Hautier et al. 2014),
neutral (Valone and Hoffman 2003, Cusson et al. 2015), and negative
(Sasaki and Lauenroth 2011) relationships. In a meta-analysis of
diversity-stability relationships, Jiang and Pu (2009) found no
significant evidence for an effect of species richness on ecosystem
stability from observational studies in terrestrial ecosystems. Thus,
there appears to be a gap between the consistency of theoretical studies
and the equivocation of empirical studies.

We argue this gap exists because the two bodies of theory that have
developed to explain species coexistence on the one hand, and
diversity-stability relationships on the other, have diverged. One
reason these two disciplines have diverged is because they have focused
on slightly different questions. Biodiversity-ecosystem stability
studies typically ask how ecosystem variability responds to different
levels of species richness at a given level of environmental variability
(reviewed in Kinzig et al. 2001, Loreau 2010), whereas coexistence
studies ask how the long term stability of species coexistence responds
to different levels of environmental variability (Chesson and Warner
1981).

To reconcile these two bodies of theory, we extend theory on the
relationship between species richness and ecosystem stability to cases
in which species coexistence explicitly depends on environmental
fluctuations and species-specific responses to environmental conditions.
We focus on the storage effect and relative nonlinearity using a general
consumer-resource model. First, we use model simulations to investigate
the diversity-stability relationship across a gradient of environmental
variability. Counter to common expectations, we find that a negative
diversity-stability relationship should be expected when sampling occurs
over natural diversity gradients and species coexistence is fluctuation
dependent. Importantly, and in line with previous theory (Ives and
Hughes 2002, Loreau 2010, Mazancourt et al. 2013), at a given level of
environmental variability, increasing species richness stabilizes
ecosystem function, even when coexistence is fluctuation dependent.

Second, we explore the net effect of increasing environmental
variability on ecosystem stability by isolating the gain in stability
due to increased richness and the loss in stability due to increased
amplitude of species fluctuations. We find that increasing environmental
variability does not always lead to a decrease in ecosystem stability:
environmental variability promotes species richness, which stabilizes
ecosystem functioning via portfolio effects even as the environment
becaomes more variable. Whether or not increasing environmental
variability results in species gains depends on the specific coexistence
mechanism, the traits of resident species, and the traits of the species
in the regional pool.

\subsection{MATERIALS AND METHODS}\label{materials-and-methods}

\subsubsection{Consumer-resource model}\label{consumer-resource-model}

We developed a semi-discrete consumer-resource model that allows many
species to coexist on one resource by either the storage effect or
relative nonlinearity. In our model, the consumer can be in one of
two-states: a dormant state \(D\) and a live state \(N\). The dormant
state could represent, for example, the seedbank of an annual plant.
Transitions between \(N\) and \(D\) occur at discrete intervals \(\tau\)
with continuous-time consumer-resource dynamics between discrete
transitions. Thus, our model is formulated as ``pulsed differential
equations'' (Pachepsky et al. 2008, Mailleret and Lemesle 2009, Mordecai
et al. 2016). For clarity we refer to \(\tau\) as years and the growing
time between years as seasons with daily (\(t\)) time steps.

During a growing season, consumer-resource dynamics are modeled as two
differential equations:

\begin{align}
\frac{\text{d}N_{i}}{\text{d}t} &= N_{i}\epsilon_if_{i}(R), \quad t \ne \tau_k\\
\frac{\text{d}R}{\text{d}t} &= - \sum\limits_{i=1,2}f_{i}(R)N_{i}, \quad t \ne \tau_k
\end{align}

\noindent where the discrete transitions between \(N\) and \(D\) occur
between seasons at times \(\tau_k\), \(k = 1,2,3, \dots, K\). The
subscript \(i\) denotes species, \(N\) is the living biomass state, and
\(\epsilon_i\) is each species' resource-to-biomass conversion
efficiency. The growth rate of living biomass is a resource-dependent
Hill function,
\(f_{i}(R) = r_{i}R^{a_{i}} / (b_{i}^{a_{i}}+R^{a_{i}})\), where
\emph{r} is a species' intrinsic growth rate and \emph{a} and \emph{b}
define the curvature of the function. Resource depletion is equal to the
sum of each species' consumption.

Along with resource uptake, consumer population growth depends on the
production of dormant biomass (\(D\)), the activation of dormant biomass
to live biomass (\(D \rightarrow N\)), and the survival of living
biomass from one year to the next. The biomass of each species' states
at the start of a growing season are equal to

\begin{align}
  D_{i}(\tau_k^+) &= (1-\gamma_{i,\tau_k})[\alpha_i N_{i}(\tau_k) + D_{i}(\tau_k)](1-\eta_i) \\
  N_{i}(\tau_k^+) &= (1-\alpha_i)N_{i}(\tau_k) + \gamma_{i,t}[\alpha_i N_{i}(\tau_k) + D_{i}(\tau_k)] (1-\eta_i),
\end{align}

\noindent where \(D(\tau_k)\), \(N(\tau_k)\), and \(R(\tau_k)\) are the
abundances of each state at the end of growing season \emph{k} and
\(\tau_k^+\) denotes the beginning of growing season \(k=1\). The
activation of dormant biomass to live biomass is controlled by
\(\gamma\), which is year (\emph{k}) and species (\emph{i}) specific.
Dormant biomass is equal to a constant fraction (\(\alpha\)) of live
biomass at the end of the previous season (\(N_{i}(\tau_k)\)), plus
survival (\(1-\eta_i\)) of dormant biomass (\(D_{i}(\tau_k)\)) at the
end of the previous year and dormant biomass remaining after live
biomass activation (\(D_{i}(\tau_k)(1-\gamma_{i,\tau_k})\)). Live
biomass is equal to newly activated dormant biomass
(\(\gamma_{i,t}[D_{i}(\tau_k)\)), minus some fraction of live biomass
that is converted to dormant biomass (\((1-\alpha_i)N_{i}(\tau_k)\)) We
assume the resource pool is not replenished within a growing season.
Resource replenishment occurs between growing seasons, and the resource
pool (\emph{R}) at the start of the growing season \emph{k}+1 is
\(R(\tau_k^+) = R^+\), where \(R^+\) is a random resource pulse drawn
from a log-normal distribution with mean \(\mu(R^+)\) and standard
deviation \(\sigma(R^+)\). Model parameters and notation are described
in Table 1.

\paragraph{Implementing the Storage
Effect}\label{implementing-the-storage-effect}

To make this a storage effect model, we need to satisfy three
conditions: (1) the organisms must have a mechanism for persistence
under unfavorable conditions, (2) species must respond differently to
environmental conditions, and (3) the effects of competition on a
species must be more strongly negative in good years relative to
unfavorable years. Our model meets condition 1 because we include a
dormant stage with very low death rates. We satisfy condition 2 with our
model whenever \(\gamma\) is not perfectly correlated between species.
Lastly, our model meets condition 3 because condition 2 partitions
intraspecific and interspecific competition into different years. Thus,
during a high \(\gamma\) year for one species, resource uptake is also
inherently high for that species, which increases intraspecific
competition relative to interspecific competition. So, given adequate
variability in \(\gamma\), the inferior competitor can persist. We
created competitive hierarchies in the storage effect version of the
model by altering species' biomass conversion efficiencies
(\(\epsilon\))

We generated sequences of (un)correlated dormant-to-live state
transition rates (\(\gamma\)) for each species by drawing from
multivariate normal distributions with mean 0 and a variance-covariance
matrix (\(\Sigma(\gamma)\)) of

\begin{align}
\Sigma(\gamma) = 
\begin{bmatrix}
\sigma^2_{E} & \rho_{1,2}\sigma^2_{E} & \rho_{1,3}\sigma^2_{E} & \rho_{1,4}\sigma^2_{E} \\
\rho_{2,1}\sigma^2_{E} & \sigma^2_{E} & \rho_{2,3}\sigma^2_{E} & \rho_{2,4}\sigma^2_{E} \\
\rho_{3,1}\sigma^2_{E} & \rho_{3,2}\sigma^2_{E} & \sigma^2_{E}  & \rho_{3,4}\sigma^2_{E} \\
\rho_{4,1}\sigma^2_{E} & \rho_{4,2}\sigma^2_{E} & \rho_{4,3}\sigma^2_{E} & \sigma^2_{E}  \\
\end{bmatrix}
\end{align}

\noindent where \(\sigma^2_{E}\) is the variance of the environmental
cue and \(\rho_{i,j}\) is the correlation between between the species
\emph{i}'s and species \emph{j}'s transition rates. \(\rho\) must be
less than 1 for stable coexistence, and in all simulations we placed
that constraint that \(\rho_{i,j} = \rho_{j,i}\) for each species pair.
The inferior competitor has the strongest potential to persist when
\(\rho=-1\) (perfectly uncorrelated transition rates). We used the
\texttt{R} function \texttt{mvrnorm} to generate sequences of
(un)correlated variates \emph{E} that we converted to germination rates
in the 0-1 range: \(\gamma = e^E / 1 + e^E\). Note that
\(\Sigma(\gamma)\) must be positive definite. So, after defining
\(\Sigma(\gamma)\) with all \(\rho_{i,j}\)s and \(\sigma^2_{E}\), we
used the \texttt{nearPD} function from the \texttt{Matrix} package in
\texttt{R} to coerce the variance-covariance matrix to be positive
definite.

\paragraph{Implementing Relative
Nonlinearity}\label{implementing-relative-nonlinearity}

When considering consumer-resource dynamics, species coexistence by
relative nonlinearity requires that each species has different nonlinear
responses to resource availability, and resource availability must
fluctuate through time. In a constant resource environment, the species
with the lowest \(R^*\) will always exclude the other species. To create
competitive hierarchies among species we manipulated species resource
uptake curves such that the species with the lowest \(R^*\) also had the
lowest maximum growth rate at high resource levels (e.g., low \emph{r},
low \emph{a}, and low \emph{b} values relative to other species in the
Hill equation; Fig. SX). Thus, our simulated species represent a
continuum from resource-conservative to resource-acquisitive. We still
allow the germination rate (\(\gamma\)) to vary, but both species are
perfectly correlated -- that is, \(\rho=1\) (Fig. 1).

\subsubsection{Numerical simulations}\label{numerical-simulations}

To understand how fluctuation-dependent coexist can affect the
diversity-stability relationship, we simulated the model with four
species under two scenarios for each coexistence mechanism. First, we
allowed the variance of the environment to determine how many species
can coexist, akin to a community assembly experiment with a species pool
of four species. This required simulating communities with all species
initially present across a gradient of annual resource variability (for
relative nonlinearity) or environmental cue variability (for the storage
effect). Second, we chose parameter values that allowed coexistence of
all four species and performed species removals. The two simulation
experiments correspond to (i) sampling ecosystem function across a
natural gradient of species richness and (ii) sampling ecosystem
function across diversity treatments within a site.

To understand how increasing environmental variability will impact
ecosystem stability when coexistence is fluctuation-dependent, we
simulated the model over a range of environmental cue variability and
species pool sizes. Thus, for each size of species pool (1, 2, 3, or 4
species), we simulated the model at 15 evenly-spaced levels of
environmental cue (range = 0.1,2) for the storage effect and 25
evenly-spaced levels of resource variability (range = 0.1,1.4) for
relative nonlinearity. We also explored the influence of species
asymmetries in competitive ability and species' correlations of
environmental responses in the storage effect model. Under the storage
effect, if all species are perfectly symmetrical, that is, there is no
superior competitor, then coexistence is fluctuation independent. We use
one such parameterization of our model to contrast the response of
ecosystem stability in fluctuation-dependent and fluctation-independent
communities to environmental variation. Likewise, under relative
nonlinearity, species' resource response curves (Fig. SX) reflect traits
that determine the intrinsic stability of each species. Therefore, we
ran two sets of simulations for relative nonlinearity: one where the
species pools increased from stable to unstable species and vice versa.
For example, if species A is the most stable species and species D is
the least stable we ran simulations where the species pool increased
from one to four species as A then B then C then D. We then ran
simulations with that order reversed.

All simulations were run for 5,000 seasons with 20-day growing seasons.
We averaged biomass over the growing season, and those yearly values
were used to calculate total community biomass in each year. After
discarding an initial 500 seasons to reduce transient effects on our
results, we calculated the coffecient of variation (\emph{CV}) of summed
species biomass through time. Therefore, in our results we refer to
ecosystem variability, which is the inverse of ecosystem stability. We
calculated species richness as the number of species whose average
biomass was greater than 1 over the course of the simulation. Parameter
values for specific results are given in figure captions. Within-season
dynamics were solved given initial conditions using the package
\texttt{deSolve} (Soetaert et al. 2010) in \texttt{R} (Team 2013). All
model code has been deposited on Figshare (\emph{link}) and is available
on GitHub at \url{http://github.com/atredennick/Coexistence-Stability}.

\subsection{RESULTS AND DISCUSSION}\label{results-and-discussion}

\subsubsection{The diversity-variability
relationship}\label{the-diversity-variability-relationship}

The direction of the diversity-variability relationship can be positive
and negative when species coexistence is maintained by
fluctuation-dependent mechanisms (Fig. 2). Ecosystem variability is
positively correlated with species richness when species richness is
measured across a gradient of environmental variability, which maintains
diversity and promotes ecosystem variability (Fig. 2A,C). If
environmental conditions are sufficient to maintain coexistence,
removing species increases ecosystem variability (Fig. 2B,D). Thus, our
results both confirm and contrast with theoretical and experimental
findings that diversity begets stability.

When we held environmental variability constant and removed species, we
produced the typical negative diversity-variability relationship (Fig.
2B,D), consistent with theoretical expecations from models with species
coexistence maintained by fluctuation-dependent mechanisms. Likewise,
our results from the species removal simulations are consistent with
results from biodiversity-ecosystem functioning experiments showing a
negative relationship between species richness and ecosystem
variability. This is encouraging because species almost certainly
coexist by some combination of fluctuation-independent (e.g., resource
partitioning) and fluctuation-dependent mechanisms. By extending theory
to communities where species richness is explicitly maintained by
temporal variability, we have gained confidence that experimental
findings are generalizable to many communities. In other words, in local
settings where environmental variability is relatively homogenous,
reductions in the number of species will reduce the stability of
ecosystem functioning, regardless of how coexistence is maintained.

When we allowed a gradient of environmental variability to determine
species coexistence, we discovered a positive relationship between
species richness and ecosystem variability (Fig. 2A,C). While surprising
when viewed through the lens of previous theory and experimental
findings, such a relationship is a direct consequence of how diversity
can be maintained in fluctuating environments. The storage effect and
relative nonlinearity both require envrionmental fluctuations to allow
niche differentiation between species pairs (Chesson 2000). Therefore,
species coexistence gains strength, for both mechanisms, as the
environment becomes more variable (Fig. SX).

Our results may explain why deviations from the negative
diversity-variability relationship often come from observational studies
(Jiang and Pu 2009). Observational studies must rely on natural
diversity gradients, and if species richness depends on environmental
variability, it is entirely possible to observe positive
diversity-variability relationships. For example, Sasaki and Lauenroth
(2011) found a negative relationship between species richness and the
temporal stability of plant abundance (a positive diversity-variability
relationship) in a semi-arid grassland. Their data came from a six sites
that were 6 km apart. While Sasaki and Lauenroth explained their results
in terms of dominant species' effects, it is also possible that each
site experienced slightly different levels of environmental variability
that influenced species coexistence. DeClerck et al. (2006) also found a
positive diversity-variability when sampling conifer richness and the
variability of productivity across a large spatial gradient in the
Sierra Nevada.

While our modeling results show that fluctuation-dependent coexistence
can create positive diversity-variability relationships, whether such
trends are detected will depend on the particular traits of the species
in the community and the relative influence of fluctuation-dependent and
fluctuation-independent coexistence mechanisms, which are not mutually
exclusive. Thus, our results may also help explain observational studies
where no relationship between diversity and variability is detected. For
example, Cusson et al. (2015) found no relationship between species
richness and variability of abundances in several marine macro-benthic
ecosystems. Many of their focal ecosystems were from highly variable
intertidal environments. If coexistence was at least in part determined
by environmental fluctuations, then the confounding effect of
variability and species richness could compensate any direct effect of
species richness on variability. Previous theoretical work showed how
environmental variation can mask the effect of species diversity on
ecosystem productivity when sampling across sites (Loreau 1998). Our
mechanistic model extends that conclusion to ecosystem stability.

\subsubsection{The impact of increasing environmental variability on
ecosystem
variability}\label{the-impact-of-increasing-environmental-variability-on-ecosystem-variability}

Whether coexistence is fluctuation-independent or fluctuation-dependent
becomes especially important when we consider how ecosystem stability
responds to increasing environmental variability. In the
fluctuation-independent case, species richness is essentially fixed
because the species' inequalities that determine coexistence (niche and
fitness differences) are not linked to environmental variability.
Therefore, increasing environmental variability will always increase
ecosystem variability by increasing the fluctuations of individual
species' abundances.

When species coexistence is fluctuation-dependent via the storage
effect, increasing environmental variability has much more interesting
effects on ecosystem variability. In Fig. 3 we show storage effect
simulation results where environmental variability determines species
coexistence from a regional species pool of four species. We also show
results from nested subsets of the four species pool (e.g., only two
species in the pool instead of four) to show the trajectory of ecosystem
variability if new species are not present to join the local community.
In accordance with coexistence theory, we find that realized species
richness increases with environmental variability and, in some cases,
increasing variability can actually completely temper the effect of
increasing environmental variability. More species rich communities are
less variable on average (e.g., lower intercepts in log-log space; Fig.
SX) and increase in ecosystem \emph{CV} at a slower rate (e.g., lower
slopes in log-log space; Fig. SX).

The dampening effect of fluctuation-dependent coexistence on increasing
environmental variability depends on the specific traits (parameter
values) of the species in the regional pool. Asymmetric competition
makes it more difficult for new species to enter the local community
(Fig. 3; compare top and bottom panels). Relatively high competition
also increases the rate at which ecosystem \emph{CV} increases with
environmental variance (Fig. SX). This is because the abundance of
inferior competitors is reduced and therefore does not influence
ecosystem \emph{CV} as much as when competition is symmetric. The
correlation of species' environmental responses also mediates the
relationship between environmental variance, species richness, and
ecosystem \emph{CV}: lower correlations make it easier for new species
to coexist and unique environmental responses are always stabilizing
(Fig. 3).

In communities where species coexist via relative nonlinearity, whether
or not the direct impact of environmental variability on ecosystem
variability is tempered by species additions depends on the species
traits of immigrating species. When additional species, which immigrate
from the regional pool, are less intrinsically stable than the resident
species, ecosystem variability increases at a constant rate even as
species are added (Fig. 4A). On the contrary, if more stable species are
added, species additions buffer the ecosystem from increasing
environmental variability (Fig. 4B). The stability of individual species
in our relative nonlinearity model is determined by their respective
resource response curves (Fig. SX). Under relative nonlinearity, we find
that the buffering effect of species additions depends on species
traits, and the order in which species enter the local community.
Indeed, if all species in the regional pool are less stable than the
resident species, then no stabilization occurs as species are added
(Fig. 4A).

Our simulation results lead to two conclusions. First, when predicting
the impacts of increasing environmental variability on ecosystem
stability, the mechanism of coexistence in the community matters.
Fluctuation-dependent coexistence can buffer ecosystems from increasing
environmental variability by allowing for species additions. As shown in
previous work (Loreau and {{de Mazancourt}} 2013), the stabilizing
effect of species additions depends on the correlations of their
environmental responses (Fig. 3e-f). Whether our theoretical predictions
hold in real communities is unknown and requires empirical tests. Doing
so would require manipulating environmental variability in communities
where coexistence is known to be fluctuation-dependent, at least in
part. Such data do exist (Angert et al. 2009), and a coupled
modeling-experimental approach could determine if our predictions hold
true in real communities.

Second, whether local fluctuation-dependent communities can receive the
benefit of additional species depends on a diverse regional species
pool. If the regional pool is not greater in size than the local species
pool, than ecosystem stability will decline with environmental
variability in a similar manner as in fluctuation-independent
communities because species richness will be fixed (Fig. 5A,B).
Metacommunity theory has made clear the importance of rescue effects to
avoid species extinctions (Brown and Kodric-Brown 1997, Leibold et al.
2004). Here, instead of local immigration by a resident species working
to rescue a species from extinction, immigration to the local community
by a new species rescues ecosystem processes from becoming less stable
(Fig. 5C,D). Thus, our results reinforce the importance of both local
and regional biodiversity conservation. Just as declines in local
species richness can destabilize ecosystem functioning (Tilman et al.
2006, Hector et al. 2010, Hautier et al. 2014), species losses at larger
spatial scales can also weaken stability. Wang and Loreau (2014) show
that regional ecosystem stability depends on regional biodiversity
through its effects on beta diversity and, in turn, the asynchrony of
functioning in local communities. Our results show that, when
coexistence is fluctuation-dependent, regional biodiversity declines
could also affect local ecosystem functioning by limiting local species
additions that could be possible under scenarios of increasing
environmental variability (Fig. 5).

Species coexistence in real ecological communities probably emerges from
some combination of fluctuation-independent and fluctuation-dependent
mechanisms (Chesson 2000, Clark et al. 2010). Likewise, environmental
conditions in real ecosystems are unlikely to change only in their
variability without an associated change in the mean (Avolio et al.
2015). Therefore, environmental change has the potential to alter the
niche and fitness differences among species in multiple ways, some of
which were not present in our current analysis. Mean changes in
environmental conditions could reorder competitive hierarchies
(Klanderud and Totland 2005) and/or alter the availability of niches
(Harpole et al. 2016). Associated changes in ecosystem stability will
depend upon the magnitude of environmental change, each species'
response to the particular environmental driver, and biotic interactions
(Hallett et al. 2014). Thus, it is becoming clear that understanding how
ecosystem stability will respond to global change will require a
trait-based approach.

\subsection{CONCLUSIONS}\label{conclusions}

How does fluctuation-dependent coexistence affect the
diversity-stability relationship? At a given level of environmental
variability, the typical positive diversity-stability relationship holds
because having more species always stabilizes ecosystem functioning.
However, counter to other theoretical studies, we found that a negative
diversity-stability relationship is also possible if sampling occurs
across a natural diversity gradient and species coexistence is dependent
on environmental fluctuations. We also found that fluctuation-dependent
species coexistence may help buffer ecosystems from increasing
environmental variability because environmental variability promotes
species richness, which, in turn, promotes stability. Where
fluctuation-dependent species coexistence prevails and environmental
variability is projected to increase, our findings suggest that
conserving regional species pools and dispersal corridors between local
communities will be important.

\subsection{ACKNOWLEDGMENTS}\label{acknowledgments}

The National Science Foundation provided funding for this work through a
Postdoctoral Research Fellowship in Biology to ATT (DBI-1400370) and a
CAREER award to PBA (DEB-1054040).

\newpage{}

\subsection{TABLE}\label{table}

\begin{table}[!htbp]
\footnotesize
\caption{Default values of model parameters and their descriptions. Parameters that vary depending on the mode and strength of species coexistence or depending on species copmetive hierarchies are labeled as "variable" in parantheses. The dormant-to-live biomass transition fraction ($\gamma$) is a function of other parameters, so has no default value.}
\begin{tabular}{l l l}
\hline
Parameter & Description & Value \\
\hline
$r$ & maximum per capita growth rate & 1 (variable) \\
$a$ & Hill function rate parameter & 2 (variable) \\
$b$ & Hill function curvature parameter & 2.5 (variable) \\
$\epsilon$ & resource-to-biomass conversion efficiency & 0.5 \\
$\alpha$ & allocation fraction of live biomass to dormant biomass & 0.5 (variable) \\
$\gamma$ & dormant-to-live biomass transition fraction & -- \\
$\rho$ & correlation of species' response to the environment & 0 (variable) \\
$\sigma_E$ & variance of the environmental cue & 2 (variable) \\
$\eta$ & dormant biomass mortality rate & 0.1 \\
$\mu(R^+)$ & mean annual resource pulse & 20 (non-log scale) \\
$\sigma(R^+)$ & standard deviation of annual resource pulse & 0 (variable) \\
\hline
\end{tabular}
\end{table}

\newpage{}

\subsection{FIGURES}\label{figures}

\begin{figure}[htbp]
\centering
\includegraphics{components/figure/manuscript-model_types-1.pdf}
\caption{Resource uptake functions and example time series of
(un)correlated germination fractions for the storage effect (A,B) and
relative nonlinearity (C,D) formulations of the consumer-resource model.
The resource uptake functions for both species are equivalent for the
storage effect, but their germination fractions are uncorrelated in
time. The opposite is true for relative nonlinearity: the two species
have unique resource uptake functions, but their germination fractions
are perfectly correlated in time.}
\end{figure}

\newpage{}

\begin{figure}[!ht]
  \centering
      \includegraphics{./components/main_figure_hires.png}
  \caption{Variability of total community as function of species richness when coexistence is maintained by the storage effect (A,B) or relative nonlinearity (C,D). Top panels show results from simulations where environmental or resource variance determine the number species that coexist in a community. Bottom panels show results from simulations where environmental or resource variance is fixed at a level that allows coexistence of all four species, but speces are removed to manipulate diversity. The top panels represent regional diversity-stability relationships across natural diversity gradients, whereas the bottom panels represent local diversity-stability relationships.}
\end{figure}

\newpage{}

\begin{figure}[!ht]
  \centering
      \includegraphics[height=5in]{./components/storage_effect_div+envar_varycomp.png}
  \caption{The effect of environmental variability on ecosystem variability with associated effects of species richness when species coexist via the storage effect. Panels (A-C) show simulation results where species have slightly asymmetrical competitive effects, whereas panels (D-F) show results when competition is more asymmetric. We show results for different levels of correlations of species' environmental responses, $\rho$.}
\end{figure}

\newpage{}

\begin{figure}[!ht]
  \centering
      \includegraphics[height=2.5in]{./components/relative_nonlinearity_div+envar.png}
  \caption{The effect of environmental variability on ecosystem variability with associated effects of species richness when species coexist via relative nonlinearity. (A) The species pool increases from 1-4 four species, with the fourth species being most unstable. Increasing environmental variability (the SD of annual resource availability) allows for greater species richness, but species additions do not modulate the effect of environmental variability on ecosystem variability. (B) The species pool increases from 1-4 four species, with the fourth species being most stable (though, the fourth species was unable to coexist under these parameter values). In this case increasing environmental variability allows for greater realized species richness and can temper the effect of environmental variability.}
\end{figure}

\newpage{}

\begin{figure}[!ht]
  \centering
      \includegraphics[height=4in]{./components/coexistence_stability_infographic_v2.png}
  \caption{Example of how species additions under increasing environmental variability can buffer ecosystem stability when species coexistence is fluctuation-dependent via the storage effect. Environmental variability ($\sigma^2_E$) increases linearly with time. (A) Time series of species' biomasses (colored lines) in a closed community where colonization of new species is not possible and (B) its associated coefficient of variation (Rolling CV; calculated over 100-yr moving window) through time. (C) Time series of species' biomasses in an open community where colonization by new species from the regional pool of four species becomes possible as environmental variation increases. The trajectory of total biomass CV in the open community (D) asymptotes at lower variability than in the closed community (B) due to the buffering effect of species richness.}
\end{figure}

\newpage{}

\setlength{\parindent}{0ex} \singlespacing

\subsection*{REFERENCES}\label{references}
\addcontentsline{toc}{subsection}{REFERENCES}

Adler, P. B., and J. M. Drake. 2008. Environmental variation, stochastic
extinction, and competitive coexistence. The American Naturalist
172:186--195.

Adler, P. B., J. HilleRisLambers, P. C. Kyriakidis, Q. Guan, and J. M.
Levine. 2006. Climate variability has a stabilizing effect on the
coexistence of prairie grasses. Proceedings of the National Academy of
Sciences 103:12793--12798.

Angert, A. L., T. E. Huxman, P. Chesson, and D. L. Venable. 2009.
Functional tradeoffs determine species coexistence via the storage
effect. Proceedings of the National Academy of Sciences of the United
States of America 106:11641--11645.

Avolio, M. L., K. J. L. Pierre, G. R. Houseman, S. E. Koerner, E. Grman,
F. Isbell, D. S. Johnson, and K. R. Wilcox. 2015. A framework for
quantifying the magnitude and variability of community responses to
global change drivers. Ecosphere 6:1--14.

Brown, J. H., and A. Kodric-Brown. 1997. Turnover Rates in Insular
Biogeography : Effect of Immigration on Extinction. Ecology 58:445--449.

C{á}ceres, C. E. 1997. Temporal variation, dormancy, and coexistence: a
field test of the storage effect. Proceedings of the National Academy of
Sciences 94:9171--9175.

Chesson, P. 2000. Mechanisms of Maintenance of Species Diversity. Annual
Review of Ecology and Systematics 31:343--366.

Chesson, P. L., and R. R. Warner. 1981. Environmental Variability
Promotes Coexistence in Lottery Competitive Systems. The American
Naturalist 117:923--943.

Chesson, P., R. L. E. Gebauer, S. Schwinning, N. Huntly, K. Wiegand, M.
S. K. Ernest, A. Sher, A. Novoplansky, and J. F. Weltzin. 2004. Resource
pulses, species interactions, and diversity maintenance in arid and
semi-arid environments. Oecologia 141:236--253.

Chesson, P., S. W. Pacala, and C. Neuhauser. 2001. Environmental Niches
and Ecosystem Functioning. Pages 213--245 \emph{in} A. P. Kinzig, S. W.
Pacala, and D. Tilman, editors. The functional consequences of
biodiversity: Empirical progress and theoretical extensions. Princeton
University Press, Princeton.

Clark, J. S., D. Bell, C. Chu, B. Courbaud, M. Dietze, M. Hersh, J.
HilleRisLambers, I. Ib{á}{ñ}ez, S. LaDeau, S. McMahon, J. Metcalf, J.
Mohan, E. Moran, L. Pangle, S. Pearson, C. Salk, Z. Shen, D. Valle, and
P. Wyckoff. 2010. High-dimensional coexistence based on individual
variation: a synthesis of evidence. Ecological Monographs 80:569--608.

Cusson, M., T. P. Crowe, R. Ara{ú}jo, F. Arenas, R. Aspden, F. Bulleri,
D. Davoult, K. Dyson, S. Fraschetti, K. Herk{ü}l, C. Hubas, S. Jenkins,
J. Kotta, P. Kraufvelin, A. Mign{é}, M. Molis, O. Mulholland, L. M.-L.
No{ë}l, D. M. Paterson, J. Saunders, P. J. Somerfield, I. Sousa-Pinto,
N. Spilmont, A. Terlizzi, and L. Benedetti-Cecchi. 2015. Relationships
between biodiversity and the stability of marine ecosystems: Comparisons
at a European scale using meta-analysis. Journal of Sea Research
98:5--14.

DeClerck, F. A. J., M. G. Barbour, and J. O. Sawyer. 2006. Species
richness and stand stability in conifer forests of the Sierra Nevada.
Ecology 87:2787--2799.

Descamps-Julien, B., and A. Gonzalez. 2005. Stable coexistence in a
fluctuating environment: An experimental demonstration. Ecology
86:2815--2824.

Elton, C. 1958. The Ecology of Invasions by Animals and Plants. Pages
1689--1699. University of Chicago Press, Chicago.

Hallett, L. M., J. S. Hsu, E. E. Cleland, S. L. Collins, T. L. Dickson,
E. C. Farrer, L. A. Gherardi, K. L. Gross, R. J. Hobbs, L. Turnbull, and
K. N. Suding. 2014. Biotic mechanisms of community stability shift along
a precipitation gradient. Ecology 95:1693--1700.

Harpole, W. S., L. L. Sullivan, E. M. Lind, J. Firn, P. B. Adler, E. T.
Borer, J. Chase, P. A. Fay, Y. Hautier, H. Hillebrand, A. S. MacDougall,
E. W. Seabloom, R. Williams, J. D. Bakker, M. W. Cadotte, E. J.
Chaneton, C. Chu, E. E. Cleland, C. D'Antonio, K. F. Davies, D. S.
Gruner, N. Hagenah, K. Kirkman, J. M. H. Knops, K. J. {La Pierre}, R. L.
McCulley, J. L. Moore, J. W. Morgan, S. M. Prober, A. C. Risch, M.
Schuetz, C. J. Stevens, and P. D. Wragg. 2016. Addition of multiple
limiting resources reduces grassland diversity. Nature 537:93--96.

Hautier, Y., E. W. Seabloom, E. T. Borer, P. B. Adler, W. S. Harpole, H.
Hillebrand, E. M. Lind, A. S. MacDougall, C. J. Stevens, J. D. Bakker,
Y. M. Buckley, C. Chu, S. L. Collins, P. Daleo, E. I. Damschen, K. F.
Davies, P. a Fay, J. Firn, D. S. Gruner, V. L. Jin, J. a Klein, J. M. H.
Knops, K. J. {La Pierre}, W. Li, R. L. McCulley, B. a Melbourne, J. L.
Moore, L. R. O'Halloran, S. M. Prober, A. C. Risch, M. Sankaran, M.
Schuetz, and A. Hector. 2014. Eutrophication weakens stabilizing effects
of diversity in natural grasslands. Nature 508:521--5.

Hector, A., Y. Hautier, P. Saner, L.Wacker, R. Bagchi, J. Joshi, M.
Scherer-Lorenzen, E. M. Spehn, E. Bazeley-White, M.Weilenmann, M. C.
Caldeira, P. G. Dimitrakopoulos, J. a. Finn, K. Huss-Danell, A.
Jumpponen, and M. Loreau. 2010. General stabilizing effects of plant
diversity on grassland productivity through population asynchrony and
overyielding. Ecology 91:2213--2220.

Ives, A. R., and J. B. Hughes. 2002. General relationships between
species diversity and stability in competitive systems. The American
naturalist 159:388--395.

Jiang, L., and Z. Pu. 2009. Different effects of species diversity on
temporal stability in single-trophic and multitrophic communities. The
American Naturalist 174:651--659.

Kinzig, A. P., S. W. Pacala, and D. Tilman (Eds.). 2001. The functional
consequences of biodiversity: Empirical progress and theoretical
extensions. Pages i--365. Princeton University Press, Princeton.

Klanderud, K., and Ø. Totland. 2005. Simulated climate change altered
dominance hierarchies and diversity of an alpine biodiversity hotspot.
Ecology 86:2047--2054.

Lehman, C. L., and D. Tilman. 2000. Biodiversity, Stability, and
Productivity in Competitive Communities. The American Naturalist
156:534--552.

Leibold, M. A., M. Holyoak, N. Mouquet, P. Amarasekare, J. M. Chase, M.
F. Hoopes, R. D. Holt, J. B. Shurin, R. Law, D. Tilman, M. Loreau, and
A. Gonzalez. 2004. The metacommunity concept: A framework for
multi-scale community ecology.

Loreau, M. 1998. Biodiversity and ecosystem functioning: a mechanistic
model. Proceedings of the National Academy of Sciences of the United
States of America 95:5632--5636.

Loreau, M. 2010. From Polutations to Ecosystems: Theoretical Fondations
for a New Ecological Synthesis.

Loreau, M., and C. {{de Mazancourt}}. 2013. Biodiversity and ecosystem
stability: A synthesis of underlying mechanisms. Ecology Letters
16:106--115.

MacArthur, R. 1955. Fluctuations of Animal Populations and a Measure of
Community Stability. Ecology 36:533--536.

Mailleret, L., and V. Lemesle. 2009. A note on semi-discrete modelling
in the life sciences. Philosophical transactions. Series A,
Mathematical, physical, and engineering sciences 367:4779--4799.

Mazancourt, C. de, F. Isbell, A. Larocque, F. Berendse, E. {De Luca}, J.
B. Grace, B. Haegeman, H. {Wayne Polley}, C. Roscher, B. Schmid, D.
Tilman, J. van Ruijven, A. Weigelt, B. J. Wilsey, and M. Loreau. 2013.
Predicting ecosystem stability from community composition and
biodiversity. Ecology Letters 16:617--625.

Mordecai, E. A., K. Gross, and C. E. Mitchell. 2016. Within-Host Niche
Differences and Fitness Trade-offs Promote Coexistence of Plant Viruses.
The American Naturalist 187:E13--E26.

Pachepsky, E., R. M. Nisbet, and W. W. Murdoch. 2008. Between discrete
and continuous: Consumer-resource dynamics with synchronized
reproduction. Ecology 89:280--288.

Sasaki, T., and W. K. Lauenroth. 2011. Dominant species, rather than
diversity, regulates temporal stability of plant communities. Oecologia
166:761--768.

Soetaert, K., T. Petzoldt, and R. W. Setzer. 2010. Package deSolve :
Solving Initial Value Differential Equations in R. Journal Of
Statistical Software 33:1--25.

Team, R. 2013. R Development Core Team. R: A Language and Environment
for Statistical Computing.

Tilman, D., P. B. Reich, and J. M. H. Knops. 2006. Biodiversity and
ecosystem stability in a decade-long grassland experiment. Nature
441:629--632.

Turnbull, L. A., J. M. Levine, M. Loreau, and A. Hector. 2013.
Coexistence, niches and biodiversity effects on ecosystem functioning.
Ecology Letters 16:116--127.

Valone, T. J., and C. D. Hoffman. 2003. A mechanistic examination of
diversity-stability relationships in annual plant communities. Oikos
103:519--527.

Wang, S., and M. Loreau. 2014. Ecosystem stability in space: \(\alpha\),
\(\beta\) and \(\gamma\) variability. Ecology Letters 17:891--901.

Yachi, S., and M. Loreau. 1999. Biodiversity and ecosystem productivity
in a fluctuating environment: the insurance hypothesis. Proceedings of
the National Academy of Sciences of the United States of America
96:1463--1468.

\end{document}
