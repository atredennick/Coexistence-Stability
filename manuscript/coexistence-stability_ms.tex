\documentclass[12pt,]{article}
\usepackage{lmodern}
\usepackage{amssymb,amsmath}
\usepackage{ifxetex,ifluatex}
\usepackage{fixltx2e} % provides \textsubscript
\ifnum 0\ifxetex 1\fi\ifluatex 1\fi=0 % if pdftex
  \usepackage[T1]{fontenc}
  \usepackage[utf8]{inputenc}
\else % if luatex or xelatex
  \ifxetex
    \usepackage{mathspec}
  \else
    \usepackage{fontspec}
  \fi
  \defaultfontfeatures{Ligatures=TeX,Scale=MatchLowercase}
\fi
% use upquote if available, for straight quotes in verbatim environments
\IfFileExists{upquote.sty}{\usepackage{upquote}}{}
% use microtype if available
\IfFileExists{microtype.sty}{%
\usepackage{microtype}
\UseMicrotypeSet[protrusion]{basicmath} % disable protrusion for tt fonts
}{}
\usepackage[margin=1in]{geometry}
\usepackage{hyperref}
\PassOptionsToPackage{usenames,dvipsnames}{color} % color is loaded by hyperref
\hypersetup{unicode=true,
            colorlinks=true,
            linkcolor=black,
            citecolor=Blue,
            urlcolor=black,
            breaklinks=true}
\urlstyle{same}  % don't use monospace font for urls
\usepackage{graphicx,grffile}
\makeatletter
\def\maxwidth{\ifdim\Gin@nat@width>\linewidth\linewidth\else\Gin@nat@width\fi}
\def\maxheight{\ifdim\Gin@nat@height>\textheight\textheight\else\Gin@nat@height\fi}
\makeatother
% Scale images if necessary, so that they will not overflow the page
% margins by default, and it is still possible to overwrite the defaults
% using explicit options in \includegraphics[width, height, ...]{}
\setkeys{Gin}{width=\maxwidth,height=\maxheight,keepaspectratio}
\IfFileExists{parskip.sty}{%
\usepackage{parskip}
}{% else
\setlength{\parindent}{0pt}
\setlength{\parskip}{6pt plus 2pt minus 1pt}
}
\setlength{\emergencystretch}{3em}  % prevent overfull lines
\providecommand{\tightlist}{%
  \setlength{\itemsep}{0pt}\setlength{\parskip}{0pt}}
\setcounter{secnumdepth}{0}
% Redefines (sub)paragraphs to behave more like sections
\ifx\paragraph\undefined\else
\let\oldparagraph\paragraph
\renewcommand{\paragraph}[1]{\oldparagraph{#1}\mbox{}}
\fi
\ifx\subparagraph\undefined\else
\let\oldsubparagraph\subparagraph
\renewcommand{\subparagraph}[1]{\oldsubparagraph{#1}\mbox{}}
\fi

%%% Use protect on footnotes to avoid problems with footnotes in titles
\let\rmarkdownfootnote\footnote%
\def\footnote{\protect\rmarkdownfootnote}

%%% Change title format to be more compact
\usepackage{titling}

% Create subtitle command for use in maketitle
\newcommand{\subtitle}[1]{
  \posttitle{
    \begin{center}\large#1\end{center}
    }
}

\setlength{\droptitle}{-2em}
  \title{}
  \pretitle{\vspace{\droptitle}}
  \posttitle{}
  \author{}
  \preauthor{}\postauthor{}
  \date{}
  \predate{}\postdate{}

\usepackage{lineno}
\linenumbers
\usepackage{setspace}
\usepackage{todonotes}
\doublespacing
\usepackage{rotating}
\usepackage{color, soul}
\usepackage[font={normalsize},labelfont={bf},labelsep=quad]{caption}
\usepackage{sectsty}
\usepackage{bm,mathrsfs}
\usepackage{mathptmx}

\begin{document}

\renewcommand\linenumberfont{\normalfont\tiny\sffamily\color{gray}}

\definecolor{blue}{rgb}{0,0,0.7} \newcommand{\new}{\textcolor{blue}}

\begin{singlespace}

\begin{centering}

\Large{\textbf{The relationship between species richness and ecosystem variability is shaped by the mechanism of coexistence}}

\bigskip{} \bigskip{}

\renewcommand*{\thefootnote}{\fnsymbol{footnote}}

\normalsize{Andrew T. Tredennick\textsuperscript{1}, Peter B. Adler\textsuperscript{1}, and Frederick R. Adler\textsuperscript{2}}

\bigskip{}

\textit{\small{\textsuperscript{1}Department of Wildland Resources and the Ecology Center, Utah State University, Logan, Utah 84322}} \\
\textit{\small{\textsuperscript{2}Departments of Biology and Mathematics, University of Utah, Salt Lake City, Utah 84112}} 

\end{centering}

\vspace{3em}

Last compile: \today

\noindent \textbf{Keywords}: coexistence, storage effect, relative nonlinearity, diversity-stability hypothesis, pulsed differential equation, consumer-resource dynamics

\noindent \textbf{Authorship}: All authors conceived the research and designed the modeling approach; ATT conducted model simulations, with input from PBA and FRA; ATT wrote the manuscript and all authors contributed to revisions.

\noindent \textbf{Data Accessibility}: There is no data associated with this manuscript. All R code necessary to reproduce our results will be archived on Figshare and released on GitHub (https://github.com/atredennick/Coexistence-Stability).

\noindent \textbf{Running Title}: Environmental variability, ecosystem variability, \& species coexistence

\noindent \textbf{Article Type}: Letter

\noindent \textbf{Number of Words in Abstract}: 150

\noindent \textbf{Number of Words in Main Text}: 4,978 (including in-text references)

\noindent \textbf{Number of References}: 48

\noindent \textbf{Number of Tables and Figures}: 1 table, 5 figures

\noindent \textbf{Corresponding Author}:  \\
\phantom{222}Andrew Tredennick  \\
\phantom{222}Department of Wildland Resources and the Ecology Center  \\
\phantom{222}Utah State University  \\
\phantom{222}5230 Old Main Hill  \\
\phantom{222}Logan, Utah 84322 USA  \\
\phantom{222}Phone: +1-970-443-1599  \\
\phantom{222}Fax: +1-435-797-3796  \\
\phantom{222}Email: atredenn@gmail.com

\end{singlespace}

\newpage{}

\begin{abstract}
Theory relating species richness to ecosystem variability typically ignores the potential for environmental variability to promote species coexistence.
Failure to account for fluctuation-dependent coexistence may explain deviations from the expected negative diversity--\new{ecosystem variability} relationship, and limits our ability to predict the consequences of increases in environmental variability.
We use a consumer-resource model to explore how coexistence via the temporal storage effect and relative nonlinearity affects ecosystem variability.
We show that a positive, rather than negative, diversity--\new{ecosystem variability} relationship is possible when ecosystem function is sampled across a natural gradient in environmental variability and diversity.
We also show how fluctuation-dependent coexistence can buffer ecosystem functioning against increasing environmental variability by promoting species richness and portfolio effects.
Our work provides a general explanation for variation in observed diversity--\new{ecosystem variability} relationships and highlights the importance of conserving regional species pools to help buffer ecosystems against predicted increases in environmental variability.
\vspace{2em}
\end{abstract}

\setlength{\parindent}{5ex}

\subsection{INTRODUCTION}\label{introduction}

MacArthur (1955), Elton (1958), and even Darwin (Turnbull et al. 2013)
recognized the potential for compensatory dynamics among species to
stabilize ecosystem functioning in fluctuating environments. This idea
underlies the ``insurance hypothesis'' (Yachi \& Loreau 1999), which
\new{states that ecosystem variability, defined as the coefficient of variation of ecosystem biomass over time, should}
decrease with diversity because species respond dissimilarly to
environmental variation, broadening the range of conditions under which
the community maintains function (Loreau 2010). A variety of theoretical
models all predict a negative relationship between species richness and
ecosystem variability (Lehman \& Tilman 2000; Ives \& Hughes 2002;
Loreau \& de Mazancourt 2013), and experimental tests tend to support
such a prediction (Tilman et al. 2006; Hector et al. 2010).

However, the ability of biodiversity--ecosystem functioning (BEF)
experiments to accurately represent real-world dynamics is debated
(Eisenhauer et al. 2016; Wardle 2016). Much of the debate centers on the
fact that BEF experimental protocols
\new{do not allow species additions from the regional pool to offset species losses in local communities.}
Theoretical work on diversity--\new{ecosystem variability} relationships
typically suffers from the same limitation: it recognizes the role of
environmental variability in driving population fluctuations which
destabilize ecosystems, but ignores the potential for environmental
variability to promote species richness and thereby help stabilize
ecosystems (Loreau 2010, but see Chesson et al. 2001).

Fluctuating environmental conditions are an important ingredient for
stable species coexistence, both in theoretical models (Chesson 2000a;
Chesson et al. 2004) and in natural communities (Cáceres 1997;
Descamps-Julien \& Gonzalez 2005; Adler et al. 2006; Angert et al.
2009). Such ``fluctuation-dependent'' coexistence
\new{emerges most easily when} species have unique environmental
responses and environmental conditions vary so that each species
experiences both favorable and unfavorable conditions, preventing
competitive exclusion (Chesson 2000a).
\new{Chesson (2000) described the two fluctuation-dependent mechanisms: the storage effect and relative nonlinearity.
Both mechanisms operate when environmental variation favors different species at different times. 
Under the storage effect, this happens because species are competing for resources at different times (and escaping competition in unfavorable periods). 
Under relative nonlinearity, all species are competing for resources at the same time, but each species alters resource availability in a way that favors its competitors.
We describe these mechanisms in more detail below (see \textbf{Materials and Methods: Consumer-resource model}).}

When coexistence is maintained by a fluctuation-dependent mechanism, an
increase in environmental variability might lead to an increase in
species richness and, consequently, a decrease in ecosystem variability.
However, increasing environmental variability may also increase
ecosystem variability by increasing the fluctuations of individual
species, regardless of species richness. These countervailing effects of
environmental variability present an interesting paradox: while we
should expect an increase in environmental fluctuations to increase
ecosystem variability, this increase might be buffered if
fluctuation-dependent coexistence adds new species to the community.
Such a paradox complicates predictions about how ecosystems will respond
to predicted departures from historical ranges of environmental
variability.

The opposing effects of environmental variability on ecosystem
variability might explain the mixed results from \new{observational}
studies on the diversity--\new{ecosystem variability} relationship.
Observational tests of the diversity--\new{ecosystem variability}
relationship, which require sampling across natural diversity gradients,
have yielded negative (Hautier et al. 2014), neutral (Valone \& Hoffman
2003; Cusson et al. 2015), and positive (Sasaki \& Lauenroth 2011)
relationships. In a meta-analysis of
diversity--\new{ecosystem variability} relationships, Jiang \& Pu (2009)
found no significant evidence for an effect of species richness on
ecosystem variability when restricting data to observational studies in
terrestrial ecosystems, perhaps because environmental variability varies
across natural diversity gradients, affecting both richness and
ecosystem variability. The idiosyncratic results of these observational
studies contrast with the consistent conclusions from
\new{experimental and theoretical work that ignore, or control, the}
feedbacks between variability and richness.

The gap between theoretical expectations and empirical results of
diversity--\new{ecosystem variability} relationships might reflect the
divergence of theory developed to explain species coexistence and theory
developed to explain diversity and ecosystem variability.
\new{In his thorough review of the topic, Loreau (2010) cautions that "one of the pieces of the stability jigsaw [puzzle] that is still missing here is the interconnection between community stability and the maintenance of species diversity due to temporal environmental variability."}
One reason these two disciplines have diverged is because they have
focused on different questions. Diversity--\new{ecosystem variability}
studies typically ask how ecosystem variability responds to different
levels of species richness at a given level of environmental variability
(reviewed in Kinzig et al. 2001; Loreau 2010), whereas coexistence
studies ask how species richness responds to different levels of
environmental variability (Chesson \& Warner 1981).

To reconcile these two perspectives, we extend theory on the
relationship between species richness and ecosystem variability to cases
in which species coexistence explicitly depends on environmental
fluctuations and species-specific responses to environmental conditions.
We focus on
\new{communities where coexistence is maintained by either the temporal storage effect or relative nonlinearity}
using a general consumer-resource model. We use the model to investigate
two questions:

\begin{enumerate}
\def\labelenumi{\arabic{enumi}.}
\item
  Does the diversity--\new{ecosystem} variability relationship remain
  negative when species coexistence is maintained by the temporal
  storage effect or relative nonlinearity?
\item
  How does increasing environmental variability impact ecosystem
  variability when coexistence depends on the storage effect or relative
  nonlinearity?
\end{enumerate}

\subsection{MATERIALS AND METHODS}\label{materials-and-methods}

\subsubsection{Consumer-resource model}\label{consumer-resource-model}

We developed a semi-discrete consumer-resource model that allows
multiple species to coexist on one resource by either the storage effect
or relative nonlinearity. In our model, the consumer can be in one of
two-states: a dormant state \(D\) and a live state \(N\). The dormant
state could represent, for example, the seed bank of an annual plant
\new{or root biomass of a perennial plant}. Transitions between \(N\)
and \(D\) occur at discrete intervals between growing seasons, with
continuous-time consumer-resource dynamics between the discrete
transitions. Thus, our model is formulated as ``pulsed differential
equations'' (Pachepsky et al. 2008; Mailleret \& Lemesle 2009; Mordecai
et al. 2016).
\new{We refer to $\tau$ as growing seasons and each growing season is composed of $T$ daily time steps, indexed by $t$ ($t=1,2,3,\dots,T$).
For example, the notation $\tau(t)$ reads as: ``day $t$ within growing season $\tau$.''}

\new{At the beginning of growing season $\tau$ a season-specific fraction ($\gamma_{i,\tau}$) of dormant biomass is activated as living biomass such that}
\vspace{-3em}

\begin{align}
N_{i,\tau(0)} = \gamma_{i,\tau} D_{i,\tau(0)},
\end{align}\vspace{-3em}

\noindent \new{where \emph{i} indexes each species and $\tau(0)$ denotes the beginning of growing season $\tau$.}
\new{Live biomass at the start of the growing season $\left(N_{i,\tau(0)}\right)$ then serves as the initial conditions for continuous-time}
consumer-resource dynamics that are modeled as two differential
equations: \vspace{-3em}

\begin{align}
\frac{\text{d}N_{i}}{\text{d}t} &= \epsilon_if_{i}(R)N_{i}, \\
\frac{\text{d}R}{\text{d}t} &= - \sum\limits_{i}f_{i}(R)N_{i},
\end{align}\vspace{-3em}

\noindent where the subscript \(i\) denotes species, \(N_i\) is living
biomass, and \(\epsilon_i\) is species-specific resource-to-biomass
conversion efficiency. The growth rate of living biomass is a
resource-dependent Hill function,
\(f_{i}(R) = r_{i}R^{a_{i}} / (b_{i}^{a_{i}}+R^{a_{i}})\), where
\emph{r} is a species' intrinsic growth rate and \emph{a} and \emph{b}
define the curvature and scale of the function, respectively. Resource
depletion is equal to the sum of consumption by all species.

\new{At the end of the growing season ($t=T$), a fraction ($\alpha_i$) of live biomass is stored as dormant biomass and a fraction of dormant biomass survives ($1-\eta_i$) to the next growing season, giving the following equation:}
\vspace{-3em}

\begin{align}
D_{i,\tau(0) + 1} = \left[\alpha_i N_{i,\tau(T)} + D_{i,\tau(T)} \right](1 - \eta_i)
\end{align}\vspace{-3em}

\noindent \new{where $\tau(T)$ denotes the end of growing season $\tau$.
We assume remaining live biomass $\left(N_{i,\tau(T)}(1-\alpha_i)\right)$ dies (i.e., this is not a closed system where all biomass must be in either $N$ or $D$ states).
We do not include extinction thresholds, or any other form of demographic stochasticity, under the assumption that we are working with abundant species with generous seed dispersal.}

We assume the resource pool is not replenished within a growing season.
Resource replenishment occurs between growing seasons, and the resource
pool (\emph{R}) at the start of the growing season is
\(R_{\tau(0)} = R^+\), where \(R^+\) is a random resource pulse drawn
from a lognormal distribution with mean \(\mu(R^+)\) and standard
deviation \(\sigma(R^+)\).
\new{Taken all together, we can combine equations 1 and 4 to define the discrete transitions between live and dormant biomass at the end of a growing season.
Thus, the initial conditions for each state ($D, N, R$) at the beginning of growing season $\tau+1$ are:}
\vspace{-3em}

\begin{align}
  D_{i,\tau(0) + 1} &= (1-\gamma_{i,\tau}) \left[\alpha_i N_{i,\tau(T)} + D_{i,\tau(T)} \right] (1-\eta_i) \\
  N_{i,\tau(0) + 1} &= \gamma_{i,\tau} \left[\alpha_i N_{i,\tau(T)} + D_{i,\tau(T)} \right] (1-\eta_i) \\
  R_{\tau(0) + 1} &= \text{lognormal}\left(\mu(R^+), \sigma(R^+) \right)_0^{200},
\end{align}\vspace{-3em}

\noindent \new{where, as above, $\tau(T)$ denotes the end of growing season $\tau$ and $\tau(0) + 1$ denotes the beginning of growing season $\tau+1$.}
\new{The subscript (0) and superscript (200) indicates a lognormal distribution truncated at those values to avoid extreme resource pulses that cause computational problems.
We used the function \texttt{urlnorm} from the \texttt{Runuran} package}
(Leydold \& Hörmann 2015)
\new{to generate values from the truncated lognormal distribution.}
Model parameters and notation are described in Table 1.

\new{Our model does not include demographic stochasticity, which can lead to stochastic extinction for small populations as environmental  variability increases}
(Boyce 1992).
\new{Previous work has shown how demographic stochasticity and coexistence mechanisms can interact to create a weak "humped-shape" relationship between coexistence time and environmental variability}
(Adler \& Drake 2008),
\new{because environmental variability increases coexistence strength and the probability of stochastic extinction simultaneously.
We do not consider this potential effect here because our focus is on large populations that would most influence ecosystem functioning.}

\new{We limit our analysis to four-species communities because it is exceedingly difficult to get more than four species to coexist via relative nonlinearity without introducing another coexistence mechanism}
(Yuan \& Chesson 2015). For consistency, we also constrain our focus to
four species communities under the storage effect, but our conclusions
apply to more species-rich communities (see Supporting Information
section SI.2).\}

\paragraph{Implementing the Storage
Effect}\label{implementing-the-storage-effect}

For the storage effect to operate, we need species-specific responses to
environmental variability, density-dependent covariance between
environmental conditions and competition (\emph{EC} covariance), and
subadditive population growth (Chesson 1994, 2000b). If these conditions
are present, all species can increase when rare and coexistence is
stable. In the storage effect, rare species increase by escaping the
effects of \emph{EC} covariance. Common species will experience greater
than average competition (\emph{C}) in environment (\emph{E}) years that
are good for them because common species cannot avoid intraspecific
competition. However, a rare species can escape intraspecific
competition and has the potential to increase rapidly in a year when the
environment is good for them but bad for the common species. \emph{EC}
covariance emerges in our model because dormant-to-live transition rates
(\(\gamma\)) are species-specific and vary through time.
\new{In a high $\gamma$ year for a common species, resource uptake will be above average because combined population size will be above average.
In a year when $\gamma$ is high for rare species and low for common species, resource uptake will be below average because combined population size will be below average.}
Subadditive population growth buffers populations against large
population decreases in unfavorable years. It is included in our model
through a dormant stage with very low death rates, which limits large
population declines in bad \emph{E} years.

We generated sequences of (un)correlated dormant-to-live state
transition rates (\(\gamma\)) for each species by drawing from
multivariate normal distributions with mean 0 and a variance-covariance
matrix (\(\Sigma(\gamma)\)) of \vspace{-3em}

\begin{align}
\Sigma(\gamma) = 
\begin{bmatrix}
1 & \rho_{1,2} & \rho_{1,3} & \rho_{1,4} \\
\rho_{2,1} & 1 & \rho_{2,3} & \rho_{2,4} \\
\rho_{3,1} & \rho_{3,2} & 1  & \rho_{3,4} \\
\rho_{4,1} & \rho_{4,2} & \rho_{4,3} & 1  \\
\end{bmatrix}
\sigma_{E}^2,
\end{align}\vspace{-2em}

\noindent where \(\sigma^2_{E}\) is the variance of the environmental
cue and \(\rho_{i,j}\) is the correlation between species \emph{i}'s and
species \emph{j}'s transition rates. \(\rho\) must be less than 1 for
stable coexistence, and in all simulations we constrained all
\(\rho_{i,j}\)'s to be equal. In a two-species community, the inferior
competitor has the greatest potential to persist when \(\rho=-1\)
(perfectly uncorrelated transition rates). However, in a four-species
community the minimum possible correlation among species is -1/3 given
our constraints that all \(\rho\)'s are equal and that
\(\Sigma(\gamma)\) must be positive-definite. We used the \texttt{R}
function \texttt{mvrnorm} to generate sequences of (un)correlated
variates \emph{E} that we converted to germination rates in the 0-1
range: \(\gamma = e^E / \left(1 + e^E \right)\).

\paragraph{Implementing Relative
Nonlinearity}\label{implementing-relative-nonlinearity}

In the absence of environmental fluctuations, the outcome of competition
between two species limited by the same resource is determined by the
shape of their resource uptake curves. That is, at constant resource
supply, whichever species has the lowest resource requirement at
equilibrium (\(R^*\)) will exclude all other species (Tilman 1982).
Resource fluctuations create opportunities for species coexistence
because the resource level will sometimes exceed the \(R^*\) of the
superior competitor. If the resource uptake curves of each species are
relatively nonlinear, then some species will be able to take advantage
of resource levels that other species cannot (Chesson 1994).

For example, in Fig. 1C we show uptake curves of two species with
different degrees of nonlinearity. Species B has the lowest \(R^*\) and
would competitively exclude species A in the absence of environmental
fluctuations. But fluctuating resource supplies can benefit species A
because it can take advantage of relatively high resource levels due its
higher saturation point. Stable coexistence is only possible, however,
if when each species is dominant it improves conditions for its
competitor. This occurs in our model because when a resource
conservative species (e.g., species B in Fig. 1C) is abundant, it will
draw resources down slowly after a pulse, and its competitor can take
advantage of that period of high resource availability. Likewise, when a
resource acquisitive species (e.g., species A in Fig. 1C) is abundant,
after a pulse it quickly draws down resources to levels that favor
resource conservative species. Such reciprocity helps each species to
increase when rare, stabilizing coexistence (Armstrong \& McGehee 1980;
Chesson 2000a; Chesson et al. 2004).

\subsubsection{Numerical simulations}\label{numerical-simulations}

To explore how fluctuation-dependent coexistence can affect the
diversity--ecosystem variability relationship, we simulated the model
with four species under two scenarios for each coexistence mechanism.
First, we allowed the variance of the environment to determine how many
species can coexist, akin to a community assembly experiment with a
species pool of four species.
\new{We simulated communities with all species initially present across a gradient of annual resource variability for relative nonlinearity (50 evenly-spaced values of $\sigma_R$ in the range [0, 1.2]) or environmental cue variability for the storage effect (100 evenly-spaced values of $\sigma_E^2$ in the range [0, 3]).}
Second, we chose parameter values that allowed coexistence of all four
species and then performed species removals
\new{at a single level of environmental variability}, akin to a
biodiversity--ecosystem function experiment. The two simulation
experiments correspond to (i) sampling ecosystem function across a
natural gradient of species richness and (ii) sampling ecosystem
function across diversity treatments within a site.
\new{We refer to the former as a "regional" relationship, and the latter as a "local" relationship.
But we do not attribute any particular area size to "region", it is simply any area over which a gradient of environmental variability exists.}

To understand how increasing environmental variability will impact
ecosystem variability when coexistence is fluctuation-dependent, we
simulated the model over a range of species pool sizes and environmental
cue or resource variability. For each size of species pool (1, 2, 3, or
4 species), we simulated the model at 15 evenly-spaced levels of
environmental cue (range = {[}0.1, 2{]}) for the storage effect and 25
evenly-spaced levels of resource variability (range = {[}0, 1.2{]}) for
relative nonlinearity. We also explored the influence of asymmetries in
species' competitive abilities and correlations in species'
environmental responses within the storage effect model.
\new{We created competitive hierarchies by making the live-to-dormant biomass fractions ($\alpha$s) unequal among species.
Small differences among values of $\alpha$ were needed to create competitive hierarchies because we chose a relatively constrained gradient of environmental cue variance.
Larger differences among values of $\alpha$ expand the region of coexistence farther along a gradient environmental cue variance.}

Under relative nonlinearity, species' resource response curves (Fig.
SI-5) reflect traits that determine the temporal variability of each
species' population growth. ``Stable'' species achieve maximum resource
uptake at low resource levels, but their maximum uptake rates are
modest. For these species, population responses to resource fluctuations
are buffered. ``Unstable'' species have very high maximum uptake rates,
which they only achieve when resource availability is high, leading to
large population fluctuations. The difference in the intrinsic stability
of these two kinds of species makes our simulations sensitive to initial
conditions. Therefore, we ran two sets of simulations for relative
nonlinearity: beginning with either stable or unstable species as a
reference point. For example, if species A is the most stable species
and species D is the least stable, we ran simulations where A then B
then C then D were \new{added to the initial pool of species}. We then
ran simulations with that order reversed.

All simulations were run for 5,000 growing seasons of 100 days each. We
averaged biomass over the growing season, and yearly values of
\new{live-state biomass} were used to calculate total community biomass
in each year. After discarding an initial 500 seasons to reduce
transient effects on our results, we calculated the coefficient of
variation (\emph{CV}) of summed species biomass through time, which
represents ecosystem variability, the inverse of ecosystem stability. We
calculated \new{realized} species richness as the number of species
whose average biomass was greater than 1 over the course of the
simulation.
\new{In some cases, realized species richness is less than number of species initialized for a simulation because of competitive exclusion.}

\new{For parameters that we did not vary, we chose values that would allow coexistence of all four species at some point along the environmental variability gradients we simulated.
Our focus is specifically on communities where fluctuation-dependent coexistence is operating, and making parameters increasingly asymmetric among species typically reduced coexistence strength or made coexistence impossible (Supporting Information section SI.3).
Changes in the absolute values of parameters also altered the strength of coexistence, but in no case did altering parameter values change our qualitative results and conclusions (Supporting Information section SI.3).}
Parameter values for specific results are given in figure captions.

Within-season dynamics were solved given initial conditions using the
package \texttt{deSolve} (Soetaert et al. 2010) in \texttt{R} (Team
2013). \texttt{R} code for our model function is in the Supporting
Information section SI.1. All model code has been deposited on Figshare
(\emph{link}) and is available on GitHub at
\url{http://github.com/atredennick/Coexistence-Stability}.

\subsection{RESULTS}\label{results}

When we allowed the variance of the environment to determine which of
four initial species coexisted, similar to a study across a natural
diversity gradient, we found a positive relationship between richness
and \new{ecosystem variability}, defined as the temporal \emph{CV} of
total community biomass (Fig. 2A,C). This was true for the storage
effect, where coexistence is maintained by fluctuating dormant-to-live
transition rates (\(\gamma\)), and for relative nonlinearity, where
coexistence is maintained by annual resource pulses. The relationship is
driven by the fact that increasing environmental variability increases
the strength of both coexistence mechanisms (Fig. SI-6). More variable
conditions promoted species richness, creating a positive relationship
between diversity and \new{ecosystem} variability.

When we performed species removals but held environmental variability at
a level that allows coexistence of all four species, similar to a
biodiversity--ecosystem functioning experiment, we found a negative
diversity--\new{ecosystem variability} relationship (Fig. 2B,D). Scatter
around the relationship was small for the storage effect because all
species have similar temporal variances. Regardless of species identity,
the presence of more species always stabilized ecosystem functioning
through portfolio effects. In contrast, scatter around the relationship
was larger for relative nonlinearity (Fig. 2D) because species with
different resource uptake curves had different population variances.
Depending on which species were present, two-species communities were
sometimes less variable than three-species communities.
\new{Furthermore, the slope of the relative nonlinearity diversity--ecosystem variability relationshp in Fig. 2D is sensitive to species' traits: the difference among species' resource uptake determines the spread of single-species communities along the y-axis.
This means that the relationship can become flat as species become more similar.}

\new{For the storage effect, total community \emph{CV} decreased with species richness at a given level of environmental variability because additional species reduced the temporal standard deviation due to portfolio effects (Fig. SI-7).
Mean biomass remained the same because all species had the same resource uptake functions, which was necessary to eliminate any potential effects of relative nonlinearity.
Portfolio effects under the storage effect remained strong in an eight-species community, where total community \emph{CV} saturated after addition of the fifth species (Fig. SI-1).
For relative nonlinearity, total community \emph{CV} decreased with species richness at a given level of environmental variability because additional species increased mean biomass (over-yielding) and, at higher richness (three to four species), reduced the temporal standard deviation (Fig. SI-7).
Mean biomass increased because some species had higher growth rates (Fig. SI-5), increasing total biomass.}

To understand how much species additions might stabilize ecosystem
functioning as environmental variability increases, we simulated our
model over a range of environmental variance and species pool sizes. For
both coexistence mechanisms, realized species richness increased with
environmental variability and, in some cases, increases in richness
completely offset the effect of moderate increases in environmental
variability on ecosystem variability (Fig. 3 and 4). More species rich
communities were less variable on average and, under the storage effect,
they increased in ecosystem \emph{CV} at a slower rate than communities
with fewer species (e.g., lower slopes in log-log space; Fig. SI-8).
\new{The buffering effect of species richness under the storage effect is also evident in Fig. 2A because the relationship between species richness and ecosystem \emph{CV} begins to saturate.
In fact, ecosystem \emph{CV} remains relatively constant past four species when species have independent responses to the environment ($\rho=0$; Fig. SI-1).}

The dampening effect of fluctuation-dependent coexistence on increasing
environmental variability depends on the specific traits (parameter
values) of the species in the regional pool.
\new{Under the storage effect, moderately asymmetric competition} makes
it more difficult for new species to enter the local community, but once
they do enter, ecosystem \emph{CV} is similar between communities with
low and moderate competitive asymmetries (Fig. 3; compare top and bottom
panels). Moderately asymmetric competition does decrease the rate at
which ecosystem \emph{CV} increases with environmental variance (Fig.
SI-8) because the abundance of inferior competitors is reduced and they
do not influence ecosystem \emph{CV} as much as when competitive
asymmetry is low. The correlation of species' environmental responses
(\(\rho\)) also mediates the relationship between environmental
variance, species richness, and ecosystem \emph{CV}: lower correlations
make it easier for new species to enter the community and contribute to
porfolio effects (Fig. 3). When the correlation of species'
environmental responses were as negative as possible (\(\rho = -1/3\)),
ecosystem \emph{CV} of the four-species community was immune to
increases in the environmental cue variance (Fig. 3A). However, more
extreme increases in the variance of the environmental cue, which
increase the number of extremely low or high germination events (i.e.,
\(\gamma \approx 0\) or 1; Fig. SI-9), eventually caused ecosystem
\emph{CV} to increase in the four species community (Fig. SI-10).

In communities where species coexist via relative nonlinearity, the
extent to which species additions buffer ecosystem stability against
increases in environmental variability depends on the species traits of
immigrating species and the order in which they enter the community.
When additional species, which immigrate from the regional pool, are
less intrinsically stable than the resident species, ecosystem
variability increases at a relatively constant rate even as species are
added (Fig. 4A; Fig. SI-11). If more stable species colonize, species
additions buffer the ecosystem from increasing environmental variability
(Fig. 4B).

\new{We tested the generality of our results under different parameters by conducting a targeted sensitivity analysis focused on parameter values and asymmetries that most affect species coexistence (Supporting Information section SI.3).
In general, altering any parameter in isolation will make coexistence easier or harder at any given level of environmental variability.
Our results are only sensitive to whether or not fluctuation-dependent coexistence is operating.}

\subsection{DISCUSSION}\label{discussion}

Theory developed for biodiversity--ecosystem function experiments
emphasizes that increases in species richness should reduce ecosystem
variability. Consistent with theoretical expectations from models in
which species coexistence is maintained by fluctuation-independent
mechanisms and with results from biodiversity--ecosystem functioning
experiments, our model of fluctuation-dependent species coexistence
(also see Chesson et al. 2001) produced a negative
diversity--\new{ecosystem variability} relationship (Fig. 2B,D). This
agreement is encouraging because
\new{empirical evidence for fluctuation-dependent coexistence is accumulating}
(Pake \& Venable 1995; Cáceres 1997; Descamps-Julien \& Gonzalez 2005;
Adler et al. 2006; Angert et al. 2009; Usinowicz et al. 2012) and
species almost certainly coexist by some combination of
fluctuation-independent (e.g., resource partitioning) and
fluctuation-dependent mechanisms (Ellner et al. 2016). By extending
theory to communities where species richness is explicitly maintained by
temporal variability, we have gained confidence that experimental
findings are generalizable to many communities. In local settings where
environmental variability is relatively homogeneous, reductions in the
number of species should increase the variability of ecosystem
functioning, regardless of how coexistence is maintained.

When we allowed communities to assemble at sites across a gradient of
environmental variability, we discovered a positive relationship between
species richness and ecosystem variability (Fig. 2A,C). While surprising
when viewed through the lens of biodiversity--ecosystem functioning
theory and experimental findings, such a relationship is predicted by
theory on coexistence in fluctuating environments. Environmental
variability is a prerequisite for the storage effect and relative
nonlinearity to stabilize coexistence (Chesson 2000a). These mechanisms
can translate increased variability into higher species richness (Fig.
SI-6), but the increase in environmental variability also increases
ecosystem variability.
\new{However, the apparent saturation of the relationship in Fig. 3A suggests that the portfolio effects that buffer ecosystems against environmental variability, and inherently emerge under the storage effect, get stronger as more species are able to coexist.
Indeed, the relationship between species richness and ecosystem \emph{CV} completely saturates under the storage effect in more species rich communities (Fig. SI-1).
This suggests neutral diversity--ecosystem variability relationships are possible due to the storage effect.}

Our results may explain why deviations from the negative
diversity--\new{ecosystem variability} relationship often come from
observational studies (Jiang \& Pu 2009). Observational studies must
rely on natural diversity gradients, which do not control for
differences in environmental variability among sites. If species
richness depends on environmental variability, it is entirely possible
to observe positive diversity--\new{ecosystem variability}
relationships. For example, DeClerck et al. (2006) found a positive
diversity--\new{ecosystem variability} when sampling conifer richness
and the variability of productivity across a large spatial gradient in
the Sierra Nevada, across which environmental variability may have
promoted coexistence. Sasaki and Lauenroth (2011) also found a positive
relationship between species richness and the temporal variability of
plant abundance in a semi-arid grassland. Their data came from a six
sites that were 6 km apart. While Sasaki and Lauenroth explained their
results in terms of dominant species' effects (e.g., Thibaut \& Connolly
2013), our findings suggest an alternative explanation: each site may
have experienced sufficiently different levels of environmental
variability to influence species coexistence.

While our modeling results show that fluctuation-dependent coexistence
can create positive diversity--\new{ecosystem variability}
relationships, whether such trends are detected will depend on the
particular traits of the species in the community and the relative
influence of fluctuation-dependent and fluctuation-independent
coexistence mechanisms. Thus, our results may also help explain
observational studies where no relationship between diversity and
variability is detected. For example, Cusson et al. (2015) found no
relationship between species richness and variability of abundances in
several marine macro-benthic ecosystems. Many of their focal ecosystems
were from highly variable intertidal environments. If coexistence was at
least in part determined by environmental fluctuations, then the
confounding effect of environmental variability and species richness
could offset or overwhlem any effect of species richness on ecosystem
variability.
\new{This may be particularly common in natural communities, where environmental fluctuations can help promote species coexistence even in cases where fluctuation-independent coexistence mechanisms are most important}
(Ellner et al. 2016). Previous theoretical work showed how environmental
variation can mask the effect of species diversity on ecosystem
productivity when sampling across sites (Loreau 1998). Our mechanistic
model extends that conclusion to ecosystem variability.

Whether coexistence is fluctuation-independent or fluctuation-dependent
becomes especially important when we consider how ecosystem variability
responds to increasing environmental variability. In the
fluctuation-independent case, species richness is essentially fixed
because the niche and fitness differences that determine coexistence are
not linked to environmental variability. Therefore, increasing
environmental variability will always increase ecosystem variability by
increasing the fluctuations of individual species' abundances. When
coexistence is fluctuation-dependent, however, the outcome is less
certain. By simulating communities with different species pool sizes
across a gradient of environmental variability, we showed that species
gains due to increasing environmental variability can buffer the direct
effect of environmental variability on ecosystem variability (Figs. 3
and 4).

\new{We relied on numerical simulations of a mechanistic model to reach our conclusions, meaning our results could be sensitive to the specific parameters values we chose.
In a targeted sensitivity analysis (Supprting Information section SI.3), we found that our qualitative results are robust so long as specific parameter combinations allow fluctuation-dependent species coexistence (by either the storage effect or relative nonlinearity).
Investigating the case in which both the storage effect and relative nonlinearity operate remains a future challenge.}

Overall, our results lead to two conclusions. First, when predicting the
impacts of increasing environmental variability on ecosystem
variability, the mechanism of coexistence matters. Fluctuation-dependent
coexistence can buffer ecosystems from increasing environmental
variability by promoting increased species richness. Whether our
theoretical predictions hold in real communities is unknown and requires
empirical tests. Doing so would require manipulating environmental
variability in communities where coexistence is known to be
fluctuation-dependent, at least in part. Such data do exist (Angert et
al. 2009), and a coupled modeling-experimental approach could determine
if our predictions hold true in natural communities.

Second, whether local fluctuation-dependent communities can receive the
benefit of additional species depends on a diverse regional species
pool. If the regional pool is not greater in size than the local species
pool, than ecosystem variability will increase with environmental
variability in a similar manner as in fluctuation-independent
communities because species richness will be fixed (Fig. 5A,B).
Metacommunity theory has made clear the importance of rescue effects to
avoid species extinctions (Brown \& Kodric-Brown 1997; Leibold et al.
2004). Here, instead of local immigration by a resident species working
to rescue a species from extinction, immigration to the local community
by a new species rescues ecosystem processes from becoming more variable
(Fig. 5C,D). Thus, our results reinforce the importance of both local
and regional biodiversity conservation. Just as declines in local
species richness can destabilize ecosystem functioning (Tilman et al.
2006; Hector et al. 2010; Hautier et al. 2014), species losses at larger
spatial scales can also increase ecosystem variability. Wang \& Loreau
(2014) show that regional ecosystem variability depends on regional
biodiversity through its effects on beta diversity and, in turn, the
asynchrony of functioning in local communities. Our results show that,
when coexistence is fluctuation-dependent, regional biodiversity
declines could also affect local ecosystem functioning by limiting local
colonization events that could be possible under scenarios of increasing
environmental variability (Fig. 5).

\subsection{ACKNOWLEDGMENTS}\label{acknowledgments}

The National Science Foundation provided funding for this work through a
Postdoctoral Research Fellowship in Biology to ATT (DBI-1400370) and a
CAREER award to PBA (DEB-1054040).
\new{The support and resources from the Center for High Performance Computing at the University of Utah are gratefully acknowledged.}
\new{We thank four anonymous reviewers for providing detailed and thoughtful comments that greatly improved the paper.}

\setlength{\parindent}{0ex} \singlespacing

\subsection*{REFERENCES}\label{references}
\addcontentsline{toc}{subsection}{REFERENCES}

\hypertarget{refs}{}
\hypertarget{ref-Adler2008}{}
Adler, P.B. \& Drake, J.M. (2008). Environmental variation, stochastic
extinction, and competitive coexistence. \emph{The American Naturalist},
172, 186--195.

\hypertarget{ref-Adler2006}{}
Adler, P.B., HilleRisLambers, J., Kyriakidis, P.C., Guan, Q. \& Levine,
J.M. (2006). Climate variability has a stabilizing effect on the
coexistence of prairie grasses. \emph{Proceedings of the National
Academy of Sciences}, 103, 12793--12798.

\hypertarget{ref-Angert2009}{}
Angert, A.L., Huxman, T.E., Chesson, P. \& Venable, D.L. (2009).
Functional tradeoffs determine species coexistence via the storage
effect. \emph{Proceedings of the National Academy of Sciences of the
United States of America}, 106, 11641--11645.

\hypertarget{ref-Armstrong1980}{}
Armstrong, R.A. \& McGehee, R. (1980). Competitive Exclusion. \emph{The
American Naturalist}, 115, 151--170.

\hypertarget{ref-Boyce1992}{}
Boyce, M.S. (1992). Population viability analysis. \emph{Annual Review
of Ecology and Systematics}, 23, 481--506.

\hypertarget{ref-Brown1997}{}
Brown, J.H. \& Kodric-Brown, A. (1997). Turnover Rates in Insular
Biogeography : Effect of Immigration on Extinction. \emph{Ecology}, 58,
445--449.

\hypertarget{ref-Caceres1997}{}
Cáceres, C.E. (1997). Temporal variation, dormancy, and coexistence: a
field test of the storage effect. \emph{Proceedings of the National
Academy of Sciences}, 94, 9171--9175.

\hypertarget{ref-Chesson2000}{}
Chesson, P. (2000a). Mechanisms of Maintenance of Species Diversity.
\emph{Annual Review of Ecology and Systematics}, 31, 343--366.

\hypertarget{ref-Chesson2000a}{}
Chesson, P. (2000b). General theory of competitive coexistence in
spatially-varying environments. \emph{Theoretical population biology},
58, 211--37.

\hypertarget{ref-Chesson2004}{}
Chesson, P., Gebauer, R.L.E., Schwinning, S., Huntly, N., Wiegand, K.,
Ernest, M.S.K., et al. (2004). Resource pulses, species interactions,
and diversity maintenance in arid and semi-arid environments.
\emph{Oecologia}, 141, 236--253.

\hypertarget{ref-Chesson2001}{}
Chesson, P., Pacala, S.W. \& Neuhauser, C. (2001). Environmental Niches
and Ecosystem Functioning. In: \emph{The functional consequences of
biodiversity: Empirical progress and theoretical extensions} (eds.
Kinzig, A.P., Pacala, S.W. \& Tilman, D.). Princeton University Press,
Princeton, pp. 213--245.

\hypertarget{ref-Chesson1994}{}
Chesson, P.L. (1994). Multispecies Competition in Variable Environments.
\emph{Theoretical Population Biology}, 45, 227.

\hypertarget{ref-Chesson1981a}{}
Chesson, P.L. \& Warner, R.R. (1981). Environmental Variability Promotes
Coexistence in Lottery Competitive Systems. \emph{The American
Naturalist}, 117, 923--943.

\hypertarget{ref-Cusson2015}{}
Cusson, M., Crowe, T.P., Araújo, R., Arenas, F., Aspden, R., Bulleri,
F., et al. (2015). Relationships between biodiversity and the stability
of marine ecosystems: Comparisons at a European scale using
meta-analysis. \emph{Journal of Sea Research}, 98, 5--14.

\hypertarget{ref-DeClerck2006}{}
DeClerck, F.A.J., Barbour, M.G. \& Sawyer, J.O. (2006). Species richness
and stand stability in conifer forests of the Sierra Nevada.
\emph{Ecology}, 87, 2787--2799.

\hypertarget{ref-Descamps-Julien2005}{}
Descamps-Julien, B. \& Gonzalez, A. (2005). Stable coexistence in a
fluctuating environment: An experimental demonstration. \emph{Ecology},
86, 2815--2824.

\hypertarget{ref-Eisenhauer2016}{}
Eisenhauer, N., Barnes, A.D., Cesarz, S., Craven, D., Ferlian, O.,
Gottschall, F., et al. (2016). Biodiversity-ecosystem function
experiments reveal the mechanisms underlying the consequences of
biodiversity change in real world ecosystems. \emph{Journal of
Vegetation Science}, 27, 1061--1070.

\hypertarget{ref-Ellner2016}{}
Ellner, S.P., Snyder, R.E. \& Adler, P.B. (2016). How to quantify the
temporal storage effect using simulations instead of math.

\hypertarget{ref-Elton1958}{}
Elton, C. (1958). \emph{The Ecology of Invasions by Animals and Plants}.
University of Chicago Press, Chicago.

\hypertarget{ref-Hautier2014}{}
Hautier, Y., Seabloom, E.W., Borer, E.T., Adler, P.B., Harpole, W.S.,
Hillebrand, H., et al. (2014). Eutrophication weakens stabilizing
effects of diversity in natural grasslands. \emph{Nature}, 508, 521--5.

\hypertarget{ref-Hector2010}{}
Hector, A., Hautier, Y., Saner, P., L.Wacker, Bagchi, R., Joshi, J., et
al. (2010). General stabilizing effects of plant diversity on grassland
productivity through population asynchrony and overyielding.
\emph{Ecology}, 91, 2213--2220.

\hypertarget{ref-Ives2002b}{}
Ives, A.R. \& Hughes, J.B. (2002). General relationships between species
diversity and stability in competitive systems. \emph{The American
naturalist}, 159, 388--395.

\hypertarget{ref-Jiang2009}{}
Jiang, L. \& Pu, Z. (2009). Different effects of species diversity on
temporal stability in single-trophic and multitrophic communities.
\emph{The American Naturalist}, 174, 651--659.

\hypertarget{ref-Kinzig2001}{}
Kinzig, A.P., Pacala, S.W. \& Tilman, D. (Eds.). (2001). \emph{The
functional consequences of biodiversity: Empirical progress and
theoretical extensions}. Princeton University Press, Princeton.

\hypertarget{ref-Lehman2000}{}
Lehman, C.L. \& Tilman, D. (2000). Biodiversity, Stability, and
Productivity in Competitive Communities. \emph{The American Naturalist},
156, 534--552.

\hypertarget{ref-Leibold2004}{}
Leibold, M.A., Holyoak, M., Mouquet, N., Amarasekare, P., Chase, J.M.,
Hoopes, M.F., et al. (2004). The metacommunity concept: A framework for
multi-scale community ecology.

\hypertarget{ref-Leydold2015}{}
Leydold, J. \& Hörmann, W. (2015). Runuran: R Interface to the UNU.RAN
Random Variate Generators.

\hypertarget{ref-Loreau1998}{}
Loreau, M. (1998). Biodiversity and ecosystem functioning: a mechanistic
model. \emph{Proceedings of the National Academy of Sciences of the
United States of America}, 95, 5632--5636.

\hypertarget{ref-Loreau2010}{}
Loreau, M. (2010). \emph{From Polutations to Ecosystems: Theoretical
Fondations for a New Ecological Synthesis}.

\hypertarget{ref-Loreau2013}{}
Loreau, M. \& de Mazancourt, C. (2013). Biodiversity and ecosystem
stability: A synthesis of underlying mechanisms. \emph{Ecology Letters},
16, 106--115.

\hypertarget{ref-MacArthur1955}{}
MacArthur, R. (1955). Fluctuations of Animal Populations and a Measure
of Community Stability. \emph{Ecology}, 36, 533--536.

\hypertarget{ref-Mailleret2009}{}
Mailleret, L. \& Lemesle, V. (2009). A note on semi-discrete modelling
in the life sciences. \emph{Philosophical transactions. Series A,
Mathematical, physical, and engineering sciences}, 367, 4779--4799.

\hypertarget{ref-Mordecai2016}{}
Mordecai, E.A., Gross, K. \& Mitchell, C.E. (2016). Within-Host Niche
Differences and Fitness Trade-offs Promote Coexistence of Plant Viruses.
\emph{The American Naturalist}, 187, E13--E26.

\hypertarget{ref-Pachepsky2008}{}
Pachepsky, E., Nisbet, R.M. \& Murdoch, W.W. (2008). Between discrete
and continuous: Consumer-resource dynamics with synchronized
reproduction. \emph{Ecology}, 89, 280--288.

\hypertarget{ref-Pake1995}{}
Pake, C.E. \& Venable, D.L. (1995). Is coexistence of Sonoran Desert
annuals mediated by temporal variability in reproductive success?
\emph{Ecology}, 76, 246--261.

\hypertarget{ref-Sasaki2011}{}
Sasaki, T. \& Lauenroth, W.K. (2011). Dominant species, rather than
diversity, regulates temporal stability of plant communities.
\emph{Oecologia}, 166, 761--768.

\hypertarget{ref-Soetaert2010}{}
Soetaert, K., Petzoldt, T. \& Setzer, R.W. (2010). Package deSolve :
Solving Initial Value Differential Equations in R. \emph{Journal Of
Statistical Software}, 33, 1--25.

\hypertarget{ref-Team2013}{}
Team, R. (2013). R Development Core Team. \emph{R: A Language and
Environment for Statistical Computing}.

\hypertarget{ref-Thibaut2013}{}
Thibaut, L.M. \& Connolly, S.R. (2013). Understanding
diversity-stability relationships: Towards a unified model of portfolio
effects. \emph{Ecology Letters}, 16, 140--150.

\hypertarget{ref-Tilman1982}{}
Tilman, D. (1982). \emph{Resource competition and community structure.}

\hypertarget{ref-Tilman2006}{}
Tilman, D., Reich, P.B. \& Knops, J.M.H. (2006). Biodiversity and
ecosystem stability in a decade-long grassland experiment.
\emph{Nature}, 441, 629--632.

\hypertarget{ref-Turnbull2013}{}
Turnbull, L.A., Levine, J.M., Loreau, M. \& Hector, A. (2013).
Coexistence, niches and biodiversity effects on ecosystem functioning.
\emph{Ecology Letters}, 16, 116--127.

\hypertarget{ref-Usinowicz2012}{}
Usinowicz, J., Wright, S.J., Ives, A.R. \& Doak, D.F. (2012).
Coexistence in tropical forests through asynchronous variation in annual
seed production. \emph{Ecology}, 93, 2073--2084.

\hypertarget{ref-Valone2003}{}
Valone, T.J. \& Hoffman, C.D. (2003). A mechanistic examination of
diversity-stability relationships in annual plant communities.
\emph{Oikos}, 103, 519--527.

\hypertarget{ref-Wang2014}{}
Wang, S. \& Loreau, M. (2014). Ecosystem stability in space: \(\alpha\),
\(\beta\) and \(\gamma\) variability. \emph{Ecology Letters}, 17,
891--901.

\hypertarget{ref-Wardle2016}{}
Wardle, D.A. (2016). Do experiments exploring plant diversity-ecosystem
functioning relationships inform how biodiversity loss impacts natural
ecosystems? \emph{Journal of Vegetation Science}, 27, 646--653.

\hypertarget{ref-Yachi1999}{}
Yachi, S. \& Loreau, M. (1999). Biodiversity and ecosystem productivity
in a fluctuating environment: the insurance hypothesis.
\emph{Proceedings of the National Academy of Sciences of the United
States of America}, 96, 1463--1468.

\hypertarget{ref-Yuan2015}{}
Yuan, C. \& Chesson, P. (2015). The relative importance of relative
nonlinearity and the storage effect in the lottery model.
\emph{Theoretical Population Biology}, 105, 39--52.

\newpage{}

\subsection{TABLE}\label{table}

\begin{table}[!htbp]
\caption{Default values of model parameters and their descriptions. Parameters that vary depending on the mode and strength of species coexistence or depending on species competitive hierarchies are labeled as ``variable'' in parentheses. The dormant-to-live biomass transition fraction ($\gamma$) is a function of other parameters, so has no default value.}
\begin{tabular}{l l l}
\hline
Parameter & Description & Value \\
\hline
$r$ & maximum per capita growth rate & 0.2 (variable) \\
$a$ & Hill function curvature parameter & 2.0 (variable) \\
$b$ & Hill function scale parameter & 2.5 (variable) \\
$\epsilon$ & resource-to-biomass conversion efficiency & 0.5 \\
$\alpha$ & allocation fraction of live biomass to dormant biomass & 0.5 (variable) \\
$\gamma$ & dormant-to-live biomass transition fraction & --- \\
$\rho$ & correlation of species' response to the environment & 0.0 (variable) \\
$\sigma_E$ & variance of the environmental cue & 2.0 (variable) \\
$\eta$ & dormant biomass mortality rate & 0.1 \\
$\mu(R^+)$ & mean annual resource pulse & 20 \\
$\sigma(R^+)$ & standard deviation of annual resource pulse & 0.0 (variable) \\
\hline
\end{tabular}
\end{table}

\newpage{}

\subsection{FIGURES}\label{figures}

\begin{figure}[htbp]
\centering
\includegraphics{components/figure/manuscript-model_types-1.png}
\caption{Resource uptake functions and example time series of
(un)correlated germination fractions for the storage effect (A,B) and
relative nonlinearity (C,D) formulations of the consumer-resource model.
The resource uptake functions for both species are equivalent for the
storage effect, but their dormant-to-live transition fractions
(\(\gamma\)) are uncorrelated in time. The opposite is true for relative
nonlinearity: the two species have unique resource uptake functions, but
their dormant-to-live transition fractions (\(\gamma\)) are perfectly
correlated in time.}
\end{figure}

\newpage{}

\begin{figure}[!ht]
  \centering
      \includegraphics[height=5in]{/Users/atredenn/Google_Drive/coexistence-stability_submission/figures/coex_stability_figure1.png}
  \caption{Variability of total community biomass as a function of species richness when coexistence is maintained by the storage effect (A,B) or relative nonlinearity (C,D). Left panels show results from simulations where environmental or resource variance determine the number species that coexist in a community. Right panels show results from simulations where environmental or resource variance is fixed at a level that allows coexistence of all four species, but species are removed to manipulate diversity. The left-hand panels represent ``regional'' diversity--ecosystem variability relationships across natural diversity gradients, whereas the right-hand panels represent ``local'' diversity--ecosystem variability relationships. \new{Note that we do not attribute any particular area size to "region", it is simply any area over which a gradient of environmental variability can emerge.} Points are jittered within discrete richness values for visual clarity. Parameter values, where species are denoted by numeric subscripts: (A) $r_1 = r_2 = r_3 = r_4 = 0.2$, $a_1 = a_2 = a_3 = a_4 = 2$, $b_1 = b_2 = b_3 = b_4 = 2.5$, $\alpha_1 = 0.5, \alpha_2 = 0.49, \alpha_3 = 0.48, \alpha_4 = 0.47$, $\rho_1 = \rho_2 = \rho_3 = \rho_4 = 0$, $\sigma_E =$ variable; (B) $r_1 = r_2 = r_3 = r_4 = 0.2$, $a_1 = a_2 = a_3 = a_4 = 2$, $b_1 = b_2 = b_3 = b_4 = 2.5$, $\alpha_1 = 0.5, \alpha_2 = 0.49, \alpha_3 = 0.48, \alpha_4 = 0.47$, $\rho_1 = \rho_2 = \rho_3 = \rho_4 = -1/3$, $\sigma_E = 4$; (C) $r_1 = 0.2, r_2 = 1, r_3 = 2, r_4 = 5$, $a_1 = 2, a_2 = 5, a_3 = 10, a_4 = 25$, $b_1 = 2.5, b_2 = 20, b_3 = 30, b_4 = 45$, $\alpha_1 = \alpha_2 = \alpha_3 = \alpha_4 = 0.5$, $\rho_1 = \rho_2 = \rho_3 = \rho_4 = 1$, $\sigma(R^+) =$ variable; (D) $r_1 = 0.2, r_2 = 1, r_3 = 2, r_4 = 5$, $a_1 = 2, a_2 = 5, a_3 = 10, a_4 = 25$, $b_1 = 2.5, b_2 = 20, b_3 = 30, b_4 = 45$, $\alpha_1 = \alpha_2 = \alpha_3 = \alpha_4 = 0.5$, $\rho_1 = \rho_2 = \rho_3 = \rho_4 = 1$, $\sigma(R^+) =$ 1.1.}
\end{figure}

\newpage{}

\begin{figure}[!ht]
  \centering
      \includegraphics[height=5in]{/Users/atredenn/Google_Drive/coexistence-stability_submission/figures/coex_stability_figure2.png}
  \caption{The effect of increasing environmental variability on ecosystem variability when species coexist via the storage effect. Panels (A-C) show simulation results where species have slightly asymmetrical competitive effects, whereas panels (D-F) show results when competition is more asymmetric. Columns show results for different levels of correlations of species' environmental responses, $\rho$. Colored vertical lines show the magnitude of environmental variability at which each level of species richness first occurs. Parameter values are as in Figure 2A except for $\alpha$s: (A-C) $\alpha_1 = 0.5, \alpha_2 = 0.495, \alpha_3 = 0.49, \alpha_4 = 0.485$; (D-F) $\alpha_1 = 0.5, \alpha_2 = 0.49, \alpha_3 = 0.48, \alpha_4 = 0.47$.}
\end{figure}

\newpage{}

\begin{figure}[!ht]
  \centering
      \includegraphics[height=2.5in]{/Users/atredenn/Google_Drive/coexistence-stability_submission/figures/coex_stability_figure3.png}
  \caption{The effect of environmental variability on ecosystem variability when species coexist via relative nonlinearity. (A) The species pool increases from one to four species, with the fourth species being most unstable (e.g., resource conservative to resource acquisitive). Increasing environmental variability (the SD of annual resource availability) allows for greater species richness, but species additions do not modulate the effect of environmental variability on ecosystem variability. (B) The species pool increases from one to four species, with the fourth species being most stable (e.g., resource acquisitive to resource conservative). In this case, increasing environmental variability allows for greater realized species richness and can temper the effect of environmental variability. Parameter values are as in Figure 2C.}
\end{figure}

\newpage{}

\begin{figure}[!ht]
  \centering
      \includegraphics[height=4in]{./components/coexistence_stability_infographic_v2.png}
  \caption{Example of how species additions under increasing environmental variability can buffer ecosystem stability when species coexist via the storage effect. Environmental variability ($\sigma^2_E$) increases linearly with time on the \emph{x}-axis. (A) Time series of species' biomass (colored lines) in a closed community where colonization of new species is prevented and (B) its associated coefficient of variation (Rolling CV; calculated over 100-yr moving window) through time. (C) Time series of species' biomass in an open community where colonization by new species from the regional pool of four species becomes possible as environmental variation increases. The trajectory of total biomass CV in the open community (D) asymptotes at lower variability than in the closed community (B) due to the buffering effect of species richness. Parameter values are as in Figure 2A except for $\alpha$s: $\alpha_1 = 0.5, \alpha_2 = 0.494, \alpha_3 = 0.49, \alpha_4 = 0.483$.}
\end{figure}


\end{document}
