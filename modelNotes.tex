\documentclass[12pt]{article}
\usepackage{graphics}
\usepackage{graphicx, verbatim}
\usepackage{amssymb}
\usepackage{amsmath}
% \usepackage[T1]{fontenc}
% \usepackage[utf8]{inputenc}
\usepackage{authblk}
\textwidth=6.2in
\textheight=8.5in
%\parskip=.3cm
\oddsidemargin=.1in
\evensidemargin=.1in
\headheight=-.3in

\usepackage{Sweave}
\begin{document}
\Sconcordance{concordance:modelNotes.tex:modelNotes.Rnw:%
1 14 1 1 0 36 1}



%To put figures in subfolder
%\SweaveOpts{prefix.string=figures/fig}
\DefineVerbatimEnvironment{Sinput}{Verbatim} {xleftmargin=2em}
\DefineVerbatimEnvironment{Soutput}{Verbatim}{xleftmargin=2em}
\DefineVerbatimEnvironment{Scode}{Verbatim}{xleftmargin=2em}

\title{\LARGE How coexistence mechanisms mediate temporal stability}
\author[1]{\large Andrew T. Tredennick}
\author[1]{\large Peter B. Adler}
\author[2]{\large Frederick Adler}
\affil[1]{\footnotesize Department of Wildland Resources and the Ecology Center, Utah State University}
\affil[2]{\footnotesize Departments of Biology and Mathematics, University of Utah}
\maketitle

%----------------------------------------------------
\section{\large Introduction}
%----------------------------------------------------
Theoretical work aimed toward identifying the mechanisms by which species richness promotes temporal stability has treated species coexistence as a foregone conclusion. In so doing, that large body of work implicitly assumes that the interaction between environmental variability and the mechanism(s) by which species coexist is trivial. However, under the same conditions of environmental variability, population dynamics will respond differently depending on the coexistence context. In turn, this leads to different dynamics at the community and ecosystem levels depending on how species interact. So, it stands to reason that identifying the key mechanisms that promote ecosystem stability requires a solid understanding of how species coexistence mediates temporal stability in fluctuating environments.

To that end, we will analyze a general consumer-resource model under different coexistence assmptions. Our starting point is a model of two plant consumers and one resource (e.g., soil moisture or nitrogen). We will focus on four cases of species coexistence:
\begin{enumerate}
  \item Relative nonlinearty
  \item Storage effect with exploitative competition only (e.g., `temporal storage effect')
  \item Storage effect with interference competition only (e.g., `spatial storage effect')
  \item A combination of all three mechanisms
\end{enumerate}

Each scenario requires different model assumptions and structure, so we will describe each in turn. Although the structure may change slightly to incorporate different coexistence mechanisms, the strength of our approach lies in the similarities among the models since we work under a unified consumer-resource framework.

\subsection{Temporal storage effect}
We start with the continuous-time model

\begin{align}
\frac{\text{d}D_{i}}{\text{d}t} &= c_{i}b_{i}N_{i} - g_{i}D_{i} - m_{D,i}D_{i}\\
\frac{\text{d}N_{i}}{\text{d}t} &= f(R,N_{i}) + g_{i}D_{i} - c_{i}b_{i}N_{i} - m_{N,i}N_{i}\\
\frac{\text{d}R_{i}}{\text{d}t} &= a(S - R) - \sum\limits_{i=1,2}f(R,N_{i})
\end{align}
where $i$ denotes species, $D$ is the dormant (long-lived) biomass state, $B$ is the living biomass (fast-growing, shorter-lived) state, $b$ is the "birth" rate of new biomass in the dormant state, $c$ is some factor related to the cost of producing the dormant state, $g$ is the time-fluctuating activation rate of dormant biomass to the live biomass state, and $m$s are biomass loss rates. The growth rate of living biomass is a resource-dependent function, $f(R,N) = r_{i}R/(K_{i}+R)$, where $r$ is the maximum growth rate and $K$ is the resource level at which growth is one-half $r$. For the resource dynamics, whose state is denoted by $R$, we use a linear resource renewal equation where $a$ scales resource turnover rate and $S$ is the resource equilibrium when consumers are absent.

To make this a "storage-effect" model, we need to satisfy three conditions: (1) the organisms must have a mechanism for persistence under unfavorable conditions, (2) species must respond differently to environmental conditions, and (3) the effects of competition on a species must be more strongly negative in good years relative to unfavorable years. Our model meets condition 1 because we include a dormant stage with very low death rates. We satisfy condition 2 with model whenever $g$ is not perfectly correlated between species. Lastly, our model meets condition 3 because condition 2 partitions intraspecific and interspecific competition into different years. So long as resource uptake rates are equal ... oops we have a problem!  





\end{document}
